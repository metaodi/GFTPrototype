% document type: science report
\documentclass[abstracton, a4paper, 12pt]{scrreprt}

% Encoding (utf8)
\usepackage[utf8]{inputenc}
\usepackage[T1]{fontenc}

% Silbentrennung (Neu-Deutsch)
\usepackage[ngerman]{babel}

% Literaturverzeichnis (Deutsch)
\usepackage{bibgerm}

% Grafiken
\usepackage[pdftex]{graphicx}
% Umfliessen von Text um Tabellen und Bilder
\usepackage{wrapfig}

% hyperlinks
\usepackage{hyperref}

% Absatz
\setlength{\parindent}{0pt} % Absatzeinzug
\setlength{\parskip}{10pt} % Absatzabstand

% Glossar
\usepackage[toc]{glossaries}
\makeglossaries

% TODO Kommentare
\usepackage{todonotes}

% Definition vom Header und Footer im Seitenlayout
\usepackage{fancyhdr} 
\pagestyle{fancy} 
\fancyhf{}

% Linien nach dem Header und vor dem Footer
%\renewcommand{\footrulewidth}{0.4pt}
%\renewcommand{\headrulewidth}{0.4pt}

\fancyhead[L]{\footnotesize{\leftmark}}
\fancyfoot[C]{\footnotesize{\thepage}}
\fancyhead[R]{}

% Header auch bei Kapitelanfangsseite
\def\chapterpagestyle{fancy}

% Tabelle
% Padding links und rechts von Zelle
\setlength{\tabcolsep}{5px}
% Padding oben und unten (mittels arraystretch)
\renewcommand{\arraystretch}{1.4}

% Syntaxhighlighter
\usepackage{listings}
\usepackage{color}
\definecolor{dkgreen}{rgb}{0,0.6,0}
\definecolor{gray}{rgb}{0.5,0.5,0.5}
\definecolor{mauve}{rgb}{0.58,0,0.82}
 
\lstset{ %
  language=HTML,                % the language of the code
  basicstyle=\footnotesize,           % the size of the fonts that are used for the code
  numbers=left,                   % where to put the line-numbers
  numberstyle=\tiny\color{gray},  % the style that is used for the line-numbers
  stepnumber=1,                   % the step between two line-numbers. If it's 1, each line will be numbered
  numbersep=5pt,                  % how far the line-numbers are from the code
  backgroundcolor=\color{white},  % choose the background color. You must add \usepackage{color}
  showspaces=false,               % show spaces adding particular underscores
  showstringspaces=false,         % underline spaces within strings
  showtabs=false,                 % show tabs within strings adding particular underscores
  frame=single,                   % adds a frame around the code
  rulecolor=\color{gray},        % if not set, the frame-color may be changed on line-breaks within not-black text (e.g. commens (green here))
  tabsize=2,                      % sets default tabsize to 2 spaces
  captionpos=b,                   % sets the caption-position to bottom
  breaklines=true,                % sets automatic line breaking
  breakatwhitespace=false,        % sets if automatic breaks should only happen at whitespace
  title=\lstname,                   % show the filename of files included with \lstinputlisting;
                                  % also try caption instead of title
  keywordstyle=\color{blue},          % keyword style
  commentstyle=\color{dkgreen},       % comment style
  stringstyle=\color{mauve},         % string literal style
  escapeinside={\%*}{*)},            % if you want to add a comment within your code
  morekeywords={*,...}               % if you want to add more keywords to the set
}


%% define toc formatting
%\usepackage[titles]{tocloft}
%\setlength{\cftsubsecindent}{3em}
%\setlength{\cftsubsecnumwidth}{3.3em}
%\setlength{\cftsubsubsecindent}{4.5em}
%\setlength{\cftsubsubsecnumwidth}{4em}

%% figure numbering
%\newcounter{myfigure}
%\renewcommand{\thefigure}{\arabic{myfigure}}
%\newcounter{mytable}
%\renewcommand{\thetable}{\arabic{mytable}}
%\usepackage{makeidx} \makeindex
%\makeglossary

%% Definition vom Seitenlayout
%\setlength{\topmargin}{-1.2cm}
%\setlength{\oddsidemargin}{0.5cm} 
%\setlength{\evensidemargin}{0.5cm}

%\setlength{\textheight}{24.5cm} 
%\setlength{\textwidth}{15cm}

%\setlength{\footskip}{1.2cm} 
%\setlength{\footnotesep}{0.4cm}

%% Neues Kapitel Makro, damit die Variablen korrekt abgefuellt werden
%\newcommand{\newchap}[1]{
%	\chapter{#1}
%	\markboth {Kapitel \thechapter.  {#1}}{Kapitel \thechapter.  {#1}}
%}
%
%% command to import a figure
%\newcommand{\fig}[5]{
%  \begin{figure}[h]
%    \begin{center}
%      \includegraphics[width=#4cm]{#1}
%    \end{center}
%    \stepcounter{myfigure}
%    \caption[#5]{#3}
%    \label{#2}
%  \end{figure}
%}
%
%\newcommand{\figtable}[4]{
%  \begin{figure}[h]
%    \begin{center}
%      {  
%			  \footnotesize
%  			\sffamily
%			  #3
%      }
%    \end{center}
%    \stepcounter{myfigure}
%    \caption[#4]{#2}
%    \label{#1}
%  \end{figure}
%}
%
%\newcommand{\tab}[4]{
%  \begin{table}[h]
%    \begin{center}
%      {  
%			  \footnotesize
%  			\sffamily
%  			\renewcommand{\arraystretch}{1.4}
%			  #3  			
%      }
%    \end{center}
%    \stepcounter{mytable}
%    \caption[#4]{#2}
%    \label{#1}
%  \end{table}
%}
%
%% command to refere to a figure
%\newcommand{\reffig}[1]{Abbildung \ref{#1}}
%
%% Definition des Nummerierungslevel
%\setcounter{secnumdepth}{4} 
%\setcounter{tocdepth}{4} 
%\setcounter{lofdepth}{1} 
%
%% Definition von Paragraphen
%\parskip=0.3cm
%\parindent=0cm
%
%% Definition vom Zeilenabstand
%\usepackage{setspace} % Zeilenabstand
%\onehalfspacing
%
%% Befehle fuer Anfuehrungszeichen
%\usepackage{xspace} % Leerschlag nach Anfuehrungszeichen
%\newcommand{\qr}{\grqq\xspace}
%\newcommand{\qrs}{\grqq\ }
%\newcommand{\ql}{\glqq}
%
%% Formatierungsbefehle zum Zitieren
%\newcommand{\page}[1]{S.~#1}
%\newcommand{\pagef}[1]{S.~#1f.}
%\newcommand{\pageff}[1]{S.~#1ff.}
%\newcommand{\pages}[2]{S.~#1--#2}

\begin{document}

% -----------------------------------------
% HEAD
% -----------------------------------------
% Titelseite
\title{Studienarbeit: Google Fusion Table Prototype}
\author{Jürg Hunziker\\jhunzike@hsr.ch
		\and
		Stefan Oderbolz\\soderbol@hsr.ch}

\date{29. Mai 2012}
\maketitle

% TODO Liste
\listoftodos

\chapter*{Impressum}
Ersteller: Jürg Hunziker, Stefan Oderbolz

\chapter*{Änderungsverlauf}
\begin{tabular}{|l|l|l|}
\hline 
Datum & Änderungen & Bearbeiter \\ 
\hline 
27.02.2012 & Dokumententwurf erstellt & Stefan Oderbolz \\ 
\hline 
01.03.2012 & Struktur erstellt & J??rg Hunziker \\ 
% revisions from git
\hline 
\end{tabular} 


% Abstract
\begin{abstract}
\todo[inline]{Abstract schreiben}
Ein Abstract ist eine rein textuelle kurze Zusammenfassung der Arbeit. Der Abstract ist für die Recherche in grossen Dokumentensammlungen geeignet. Er umfasst nie mehr als eine Seite, typisch sogar nur etwa 200 Worte (etwa 20 Zeilen). 
Der Begriff 'Kurzfassung' ist zuwenig genau definiert; er soll wenn möglich vermieden werden.
\end{abstract}

% Management Summary / Web-Publikation
\chapter*{Management Summary und Web-Publikation}
\todo[inline]{Management Summary erstellen}
Das Management Summary soll 2-5 Seiten umfassen sowie eine bis zwei Figuren enthalten. Es richtet sich an den „gebildete Laien“ auf dem Gebiet und beschreibt daher in erster Linie die (neuen und eigenen) Ergebnisse und Resultate der Arbeit. Die Sprache soll knapp, klar und stark untergliedert sein. 
Grundlage für das Management Summary kann der Broschüren-Eintrag sein, den die Abteilung bei Diplomarbeiten jeweils früh verlangt, um eine Broschüre zu drucken. Das Management Summary dient als Vorlage für eine allfällige Web-Publikation.
Das Abstract  und das Management Summary werden - zeitlich gesehen - gegen Schluss der Arbeit geschrieben und bilden zusammen mit den Schlussfolgerungen im technischen Bericht den am häufigsten gelesenen Teil der Arbeit. Diese Dokumente sollen daher am Sorgfältigsten ausgearbeitet sein.
Die folgenden Stichworte sollen die typische Struktur illustrieren, wobei die genaue Ausführung jeweils auf die spezifischen Bedürfnisse und Randbedingungen eines Projekts anzupassen ist. Diese Struktur kann auch für die Präsentation der Arbeit als "Richtschnur" dienen. 
\begin{enumerate}
\item Ausgangslage
	\begin{itemize}
 		\item Warum machen wir das Projekt?
		\item Welche Ziele wurden gesteckt (Kann-Ziele, Muss-Ziele)
		\item Was machen andere / welche ähnlichen Arbeiten gibt es zum Thema?
		\item Vorgehen: Was wurde gemacht? In welchen Teilschritten?
		\item Risiken der Arbeit?
		\item Wer war involviert (Durchführung, Entscheide usw.)?
		\item Was konnte von anderen verwendet werden?
	\end{itemize}
\item Ergebnisse
	\begin{itemize}
 		\item Was ist das Resultat? 
 		\item Bewertung der Resultate, was ist Neuartig an der Arbeit?
 		\item Zielerreichung bezüglich Kann-/Muss-Zielen
 		\item Abweichungen (positiv und negativ) und kurze Begründung dafür (Externe) Kosten der Arbeit?
 		\item Was ist der Nutzen (quantifizierbar/nicht quantifizierbar)?
	\end{itemize}
\item Ausblick
	\begin{itemize}
 		\item Was hat man mit Durchführung des Projekts gelernt?
 		\item Verbleibende Probleme, (zukünftige) Gegenmassnahmen bez. Risiken
 		\item Was würde man anders machen, was ist weiter zu tun
 	\end{itemize}
\end{enumerate}

% Inhaltsverzeichnis
\tableofcontents

% -----------------------------------------
% BODY
% -----------------------------------------
% Einleitung
\part{Einleitung}
\chapter{Einleitung}
In unserer Arbeit untersuchen wir die Möglichkeiten welche die Clouddatenbank Google Fusion Tables bietet. Es sollen einige Prototypen für verschiedenste Anwendungsfälle im GIS-Bereich erstellt werden, welche das Potential der Datenbank aufzeigen.

Die Aufgabenstellung stammt von der GEOINFO AG, welche massgeschneiderte GIS-
Softwarelösungen für ihre Kunden entwickelt.

Für solche Unternehmen wird es nach und nach schwieriger sich auf dem Markt zu beweisen, da bereits viele cloudbasierte GIS-Lösungen sehr günstig oder gar kostenlos erhältlich sind. Durch die Prototypen soll ersichtlich gemacht werden, welche Anwendungsfälle von bestehenden proprietären GIS-Sytemen bereits mit Google Fusion Tables realisierbar wären und welchen Aufwand dies darstellen würde.

\section{Problemstellung}
Die Arbeit umfasst zwei Themenbereiche. Einen theoretischen Teil in dem untersucht wird, inwiefern Google Fusion Tables eine Konkurrenz für bestehende proprietäre GIS-Lösungen darstellt. Als zweiten Teil sollen verschiedene Anwendungsfälle mit Google Fusion Tables als Prototypen nachgebaut werden, wobei einer davon als mobile Webapplikation ausprogrammiert werden soll.

\subsection{Potential von Google Fusion Tables}
In einem theoretischen Teil sollen Erkenntnisse gewonnen werden, inwiefern sich bestehende GIS-Lösungen mit Google Fusion Tables ablösen lassen.

Dies soll am Beispiel eines Migrationsszenarios festgestellt werden. Die bestehenden GIS-Daten sollen möglichst einfach Exportiert und anschliessend in eine Google Fusion Tables-Datenbank importiert werden. Dafür werden die geeigneten Formate zur Übertragung der Daten evaluiert.

Da sich die Datenstrukturen von bestehenden Lösungen stark unterscheiden können, ist es nicht das Ziel eine Schritt-für-Schritt Anleitung zu erstellen. Primär geht es darum verschiedene Migrationslösungen zu entwickeln und untereinander zu vergleichen.

Dadurch soll das gegenwärtige Potential von Google Fusion Tables abgeschätzt werden. Diese Erkenntnis soll GIS-Lösungsprovidern einen Überblick verschaffen, inwiefern Google Fusion Tables bereits als einen möglichen Konkurrenten angesehen werden muss.

\section{Ziele}
We hope you will enjoy using this release as much as we enjoyed creating it. If you have comments, suggestions or wish to report an issue you are experiencing - contact us at: \emph{http://gummi.midnightcoding.org}.

\section{Rahmenbedingung}

\section{Vorgehen}

\section{Aufbau der Arbeit}

\section{Stand der Technik}

\section{Bewertung (Evaluation)}

\section{Vision/Umsetzungskonzept}

\section{Resultate der Arbeit}

\section{Schlussfolgerungen}

\section{Ausblick}

% Projektdokumentation
\part{Projektdokumentation}
% Einführung in Google Fusion Tables
\chapter{Einführung in Google Fusion Tables}

\section{Was ist Google Fusion Tables}
Google Fusion Tables wurde am 10. Juni 2009 der Öffentlichkeit zugänglich gemacht\cite{fusion-table-announce}. Das erklärte Ziel dabei war es, die Nutzung einer Datenbank so einfach wie möglich zu machen.

\begin{figure}[h]
	\centering
	\includegraphics[width=450px]{images/einfuehrung/gft-webgui-table}
	\caption{Tabellenansicht von Google Fusion Tables im Web UI}
	\label{gft-webgui-table}
\end{figure}

\begin{figure}[h]
	\centering
	\includegraphics[width=450px]{images/einfuehrung/gft-webgui-map}
	\caption{Kartenansicht von Google Fusion Tables im Web UI}
	\label{gft-webgui-map}
\end{figure}

\subsection{Software-as-a-Service (SaaS) / Cloud}
Der Begriff \emph{Software-as-a-Service} (Saas) hat sich in den letzten Jahren etabliert und bezeichnet die Dienstleistung eine Software nicht nur für einen Kunden zu entwickeln, sondern auch gleich deren Betrieb zu übernehmen. Diese gesamthafte Dienstleistung wird dann dem Kunden angeboten, so dass dieser keine eigene Infrastruktur betreiben muss. Die "`Cloud"' ist die logische  Erweiterung dieses Konzepts, dabei wird der angebotene Dienst transparent auf mehreren Umgebungen und an verschiedenen Lokationenn angeboten. Dies soll zum einen eine hohe Erreichbarkeit gewährleisten, zum anderen kann ein Anbieter dadurch sehr leicht skalieren. \todo[inline]{Quellenverweis}

\subsection{Datenbank in der Cloud}
Google Fusion Tables schafft das Problem der Erreichbarkeit einer Datenbank ab. Fusion Tables sind dezentral in der Cloud gespeichert und dort lassen sie sich einfach vertikal skalieren. Die momentanen geltenden Limiten der Datenbank sind 250MB Speicher für eine Tabelle, 25'000 Abfragen pro Tag und Benutzer sowie 100'000 Elemente, die gleichzeitig auf der Karte dargestellt werden können. Allgemein werden bei Abfragen nur die ersten 100'000 Resultate als Antwort zurückgeliefert. Diese Einschränkungen können Kunden mit Google Maps Premier auf Anfrage verändern. \cite{fusion-tables-geo-limits}

\subsection{Kollaboration und Nutzung von (ver-)öffentlichen Daten}
Der Gedanke der Cloud lässt sich abseits vom technischen noch weiterdenken. Durch die allgemeine Verfügbarkeit der Fusion Tables Datenbank, lassen sich die dort eingetragenen Daten mit anderen Benutzern oder gar der Öffentlichkeit teilen. Es gibt sogar eine eigene Suche für öffentliche Tabellen.\cite{fusion-tables-search} Tabellen können dann mit anderen Tabellen gemerged werden, wodurch die Vereinigung der Daten wiederum neue Möglichkeiten zur Nutzung der Daten ermöglicht. Eine öffentlich zugängliche Tabelle mit Länderpolygonen lässt sich so beliebig oft gebrauchen, um Daten mit Ländern zu verknüpfen und diese dann auf einer Karte darzustellen.

Diese passive Kollaboration ermöglicht es auf eine breite Palette an öffentlichen bzw. veröffentlichten Daten zurückzugreifen. Via Import lassen sich auch andere bereits bestehende Daten mit Fusion Tables nutzen. Unterstützt sind dabei Tabellen (via Google Spreadsheet) und KML Dateien. Mit unserem im Kapitel \ref{converter-build} vorgestellten Build lassen sich eine breite Palette von Dateinformaten in Google Fusion Tables importieren.

Um aktiv zu kollaborieren bieten hat Google auch einige Features angedacht. So ist es möglich einzelne Records oder gar Zellen in der Tabelle zu kommentieren, sofern man die nötigen Berechtigungen dafür hat. Falls man Daten von mehreren Lieferanten verwalten lassen will, kann eine Partei eine Tabelle erstellen und verschiedene Views darauf erstellen, welche jeweils von den verschiedenen Datenlieferanten gepflegt werden. Durch entsprechend gesetzte Berechtigungen kann so jeder seine Daten zum ganzen Beitragen, ohne Zugriff auf die Daten aller anderen Lieferanten zu erlangen. Einzig der Besitzer der Tabelle hat den kompletten Überblick. Google hat als Beispiel für diesen Use Case die Applikation \emph{Flu Vaccine Finder} erstellt, welche es den Anbietern von Grippeimpfungen ermöglicht ihre Lokale selbstständig zu erfassen und verwalten.\cite{data-gathering}

\subsection{Verschiedene Tabellenarten}
\todo[inline]{Abschnitt zu Tabellenarten}
Google Fusion Tables unterscheidet grundsätzlich 3 verschiedene Tabellenarten. Jede Tabelle hat sowohl einen numerischen wie auch einen \emph{verschlüsselten} Namen, der \emph{encid} genannt wird. Wobei das neue API nur noch die encid akzeptiert (siehe Abschnitt \ref{api-migration}). 

\subsubsection{Table}

\subsubsection{Merged Table}

\subsubsection{View}

% SQL API
\section{SQL API}
\label{sql-api}
Das SQL API bietet eine Schnittstelle mit welcher man mit SQL-ähnlichen Befehlen Daten aus Google Fusion Tables abfragen oder verändern kann. Sie verfügt bereits über eine grosse Palette an möglichen Befehlen\footnote{Befehlsreferenz: \url{https://developers.google.com/fusiontables/docs/developers_reference}}. Die SQL-Befehle werden als Parameter in folgender Form an das API übergeben:

\url{https://www.googleapis.com/fusiontables/<apiVersion>/query?sql=<statement>}

Lesende Zugriffe (\inlinecode{SELECT}, \inlinecode{SHOW TABLES}, \inlinecode{DESCRIBE}) werden dabei als \inlinecode{GET}-Request geschickt, schreibende Zugriffe (\inlinecode{CREATE}, \inlinecode{DROP}, \inlinecode{INSERT}, \inlinecode{UPDATE}, \inlinecode{DELETE} mit der \inlinecode{POST}-Methode. Um Daten zu schreiben und für den Zugriff auf private Tabelle ist eine Authentifizierung (siehe Abschnitt \ref{oauth}) mit OAuth nötig.

\begin{longtable}{|p{0.2\twocelltabwidth}|p{0.8\twocelltabwidth}|}
\hline 
\textbf{Befehl} & \textbf{Beschreibung} \\ 
\hline 
\inlinecode{SHOW TABLES} & Abfrage aller Tabellen des angemeldeten Benutzers \\ 
\hline 
\inlinecode{DESCRIBE} & Bezeichnung und Datentypen aller Spalten in einer Tabelle \\ 
\hline 
\inlinecode{CREATE TABLE} & Erstellen einer neuen Tabelle \\ 
\hline 
\inlinecode{CREATE VIEW} & Erstellen einer View auf Grundlage einer bestehenden Tabelle \\ 
\hline 
\inlinecode{SELECT} & Selektieren von Daten einer Tabelle \\ 
\hline 
\inlinecode{INSERT} & Neue Zeile zu einer Tabelle hinzufügen \\ 
\hline 
\inlinecode{UPDATE} & Daten in einer Tabelle verändern \\ 
\hline 
\inlinecode{DELETE} & Daten aus einer Tabelle löschen \\ 
\hline 
\inlinecode{DROP TABLE} & Löschen einer Tabelle \\ 
\hline 
\caption{Liste der verfügbaren SQL Befehle des SQL APIs}
\end{longtable}

\subsection{Abfragen (Queries)}
Mit dem SQL API lassen sich Abfragen an Google Fusion Tables stellen. Die untenstehende Tabelle gibt eine Übersicht der Möglichkeiten.

\begin{longtable}{|p{0.25\twocelltabwidth}|p{0.75\twocelltabwidth}|}
\hline 
\textbf{Bereich} & \textbf{Beschreibung} \\ 
\hline
Aggregation &  Folgende Funktionen sind unterstützt:
\begin{itemize}[noitemsep]
\item \inlinecode{COUNT( )}
\item \inlinecode{SUM ( {\textless}column{\_}name{\textgreater} )}
\item \inlinecode{AVERAGE ( {\textless}column{\_}name{\textgreater} )}
\item \inlinecode{MAXIMUM ( {\textless}column{\_}name{\textgreater} )}
\item \inlinecode{MINIMUM ( {\textless}column{\_}name{\textgreater} )}
\end{itemize} \\ 
\hline 
Einschränkung auf Datensatz &  \inlinecode{ROWID = {\textless}id{\textgreater}} \\
\hline 
Einschränkung auf Attributen &  Für Zahlen: \inlinecode{{\textless}column{\_}name{\textgreater} {\textless}operator{\textgreater} {\textless}number{\textgreater}}
wobei \inlinecode{{\textless}operator{\textgreater}} folgende Werte haben kann: 
\begin{itemize}[noitemsep]
\item \inlinecode{{\textgreater}, {\textless},{\textgreater}=, {\textless}=, =}
\end{itemize}

Für Strings: \inlinecode{{\textless}column{\_}name{\textgreater} {\textless}operator{\textgreater} {\textless}string{\textgreater} }
wobei \inlinecode{{\textless}operator{\textgreater}} folgende Werte haben kann: 
\begin{itemize}[noitemsep]
\item \inlinecode{\textgreater, {\textless}, \textgreater=, {\textless}=, =}
\item \inlinecode{LIKE}
\item \inlinecode{MATCHES}
\item \inlinecode{STARTS WITH}
\item \inlinecode{ENDS WITH}
\item \inlinecode{CONTAINS}
\item \inlinecode{CONTAINS IGNORING CASE}
\item \inlinecode{DOES NOT CONTAIN}
\item \inlinecode{NOT EQUAL TO}
\item \inlinecode{IN}
\end{itemize} \\
\hline
Einschränkung auf Anzahl Datensätze & \inlinecode{LIMIT {\textless}number{\textgreater}} 
wobei \inlinecode{{\textless}number{\textgreater}} angibt wieviele Datensätze des Resultats zurückgeliefert werden sollen.


Mit \inlinecode{OFFSET {\textless}number{\textgreater}} kann der Bereich, ab welchem das Limit zählt verändert werden.
\\
\hline 
\caption{Liste der verfügbaren Abfragemöglichkeiten des SQL APIs}
\end{longtable}

\subsection{Ortsbezogene Abfragen (Spatial-Queries)}
\label{sqlapi-spatialqueries}
Das SQL API bieten zudem eine Reihe von speziellen ortsabhängigen Abfrage-Möglichkeiten, welche in der folgenden Tabelle dokumentiert sind.

\begin{longtable}{|p{0.25\twocelltabwidth}|p{0.75\twocelltabwidth}|}
\hline 
\textbf{Spatial Keyword} & \textbf{Beschreibung} \\ 
\hline 
\inlinecode{ST{\_}INTERSECTS( {\textless}location{\_}column{\textgreater}, {\textless}geometry{\textgreater} )} & Kann als Bedingung in der \inlinecode{WHERE}-Klausel des Statements verwendet werden.

Liefert alle Zeilen zurück, welche sich innerhalb der definierten Geometrie \inlinecode{{\textless}geometry{\textgreater}} befinden.

\begin{itemize}[noitemsep]
\item Als \inlinecode{{\textless}location{\_}column{\textgreater}} muss eine Spalte der Tabelle angegeben werden, welche den Typ \emph{Location} hat.
\item Als \inlinecode{{\textless}geometry{\textgreater}} kann entweder ein \inlinecode{CIRCLE} oder ein \inlinecode{RECTANGLE} verwendet werden. 
\end{itemize}

\textit{Hinweis: \inlinecode{ST{\_}INTERSECTS} und \inlinecode{ST{\_}DISTANCE} dürfen nicht zusammen im gleichen Statement verwendet werden.} \\ 
\hline 
\inlinecode{ST{\_}DISTANCE( {\textless}location{\_}column{\textgreater}, {\textless}coordinate{\textgreater} )} & Kann als Bedingung in der \inlinecode{ORDER BY}-Klausel des Statements verwendet werden.

Liefert die Datensätze sortiert nach der Distanz zur angegebenen Koordinate \inlinecode{{\textless}coordinate{\textgreater}} zurück.

\begin{itemize}[noitemsep]
\item Als \inlinecode{{\textless}location{\_}column{\textgreater}} muss eine Spalte der Tabelle angegeben werden, welche den Typ \emph{Location} hat.
\item Die \inlinecode{{\textless}coordinate{\textgreater}} stellt die Koordinate dar, zu welcher der Abstand gemessen werden soll. 
\end{itemize}

\textit{Hinweis: \inlinecode{ST{\_}INTERSECTS} und \inlinecode{ST{\_}DISTANCE} dürfen nicht zusammen im gleichen Statement verwendet werden.} \\ 
\hline 
\inlinecode{CIRCLE( {\textless}coordinate{\textgreater}, {\textless}radius{\textgreater} )} & Wird verwendet, um einen Kreis von der angegebenen Koordinate \inlinecode{{\textless}coordinate{\textgreater}} mit den Radius \inlinecode{{\textless}radius{\textgreater}} zu erhalten. \\ 
\hline 
\inlinecode{POLYGON( {\textless}coordinate{\_}1{\textgreater}, {\textless}coordinate{\_}2{\textgreater}, ... )} & Wird verwendet um ein Polygon bestehend aus den angegebenen Koordinaten \inlinecode{{\textless}{coordinate{\_}x}{\textgreater}} zu erhalten. \\ 
\hline 
\inlinecode{RECTANGLE( {\textless}coordinate{\_}1{\textgreater}, {\textless}coordinate{\_}2{\textgreater} )} & Wird verwendet um ein Rechteck mit den Ecken \inlinecode{{\textless}coordinate{\_}1{\textgreater}} (links oben) und \inlinecode{{\textless}coordinate{\_}2{\textgreater}} (rechts unten) zu erhalten. \\ 
\hline 
\caption{Liste der verfügbaren Spatial Queries des SQL APIs}
\end{longtable} 

\section{Client Libraries}
Google bietet zum API bereits auch Client Libraries in den Sprachen PHP und Phyton an. Da unserer Applikation aber möglichst nur in Javascript implementiert werden soll erstellten wir uns eine Javascript Library zur Verwendung des SQL APIs.

\todo[inline]{Client Libraries Abschnitt überarbeiten (Glossareintrag für SOP, JSONP)}
Durch die Same origin policy\footnote{Die Same-Origin-Policy (SOP) ist ein Sicherheitskonzept, das es JavaScript und ActionScript nur dann erlaubt, auf Objekte einer anderen Webseite zuzugreifen, wenn sie aus derselben Quelle (Origin) stammen.\cite{sop} }, welche es uns daran hinderte AJAX-Requests direkt auf das Google API abzusetzen, mussten wir zuerst nach Lösungen für dieses Problem suchen. Wir wollten es verhindern einen PHP-Server dazwischen zu schalten, welcher uns die Abfragen abnimmt.

So fanden wir in den Google Groups ein inoffizielles JSONP API, welches es erlaubt AJAX-Requests auch über die eigene Domäne hinweg zu senden. Dies funktioniert jedoch nur für lesende Zugriffe. Für alle schreibenden Zugriffe mussten wir eine Umgeheungslösung bauen, bei der die Requests über unseren Webserver geschleust wurden. Damit konnten wir die Same-Origin-Policy umgehen. Mit Hilfe des Trusted Tester API (Siehe Abschnitt~\ref{trusted-tester-api}) haben wir schlussendlich dann aber doch noch eine Lösung gefunden, welche direkt im Browser läuft und somit nicht auf Server-Code angewiesen ist.

\section{Trusted Tester API}
\label{trusted-tester-api}
Es gibt derzeit zwei verschiedene Versionen des APIs: eine frei öffentlich zugängliche und das sogenannte \emph{Trusted Tester API}, welche derzeit im Beta-Stadium ist und ausgewählten Personen zur Verfügung steht. Als wir im Sprint 2 davon erfahren haben, haben wir uns für einen Zugang beworben, welchen wir dann auch erhalten haben.

Das Trusted Tester API bietet einige Neuerungen zur alten Schnittstelle. Es handelt sich bei diesem API um eine Vorabversion der neuen Schnittstelle, welche zukünftig ebenfalls der Öffentlichkeit zur Verfügung stehen soll. Neben dem API gibt es auch eine zugehörige Mailingsliste, auf der die Entwickler von Google Fragen beantworten und Tipps geben, aber auch dazu aufrufen, dass neue API ausgiebig zu testen.

Zu den Neuerungen des neuen APIs gehört, dass es auch eine \gls{REST}-Schnittstelle bekommen hat. Damit ist es zum einen möglich Tabelleninformationen abzufragen als auch ganz klassisch CRUD-Operationen auf dem API auszuführen. Da wir uns bemüht haben unsere Beispiele möglichst ohne Server-Code zu schreiben, konnten wir besonders davon profitieren, mit dem neuen API \inlinecode{POST}-Requests abzuschicken und somit endlich auch Schreiboperationen direkt via Browser zu unterstützen.

Alle Request an das \gls{REST}-API haben die folgende Form: \\
\url{https://www.googleapis.com/fusiontables/<apiVersion>/<resourcePath>?<parameters>}

Folgende Ressourcen sind derzeit unterstützt:
\begin{itemize}
	\item Table
	\item Column
	\item Template
	\item Style
\end{itemize}

Eine Row oder Query Ressource fehlt noch. Um Abfragen zu machen muss nach wie vor das SQL API (siehe Abschnitt \ref{sql-api}) verwendet werden.

\begin{longtable}{|p{0.5\twocelltabwidth}|p{0.5\twocelltabwidth}|}
\hline 
\textbf{Ziel} & \textbf{HTTP Mapping} \\ 
\hline 
Alle Ressourcen eines Typs auflisten & \inlinecode{GET} auf einem Ressource-Typ\\ 
\hline 
Eine spezifische Ressource holen	& \inlinecode{GET} auf einer Ressource\\
\hline 
Eine neue Ressource einfügen (kreiert eine neue Ressource) & \inlinecode{POST} auf einem Ressource-Typ (mit Daten um eine neue Ressource zu erstellen)\\
\hline 
Aktualisieren einer bestehenden Ressource & \inlinecode{PUT} auf einer Ressource (mit Daten um die Ressource zu aktualisieren)\\
\hline 
Löschen einer Ressource & \inlinecode{DELETE}  auf einer Ressource\\
\hline 
\caption{HTTP Mapping des REST-APIs}
\end{longtable}


% Geocodierung
\section{Geocodierung in Google Fusion Tables}
\label{gft-geocoding}
Ein grosser Vorteil der Google Fusion Tables ist die automatische 
\gls{Geocodierung} von Standortdaten. Sobald eine neue Zeile via Web-GUI zu einer Tabelle hinzugefügt wird, werden alle Zellen vom Typ \emph{Location} einem eindeutigen Standort auf der Karte zugewiesen. Ist dies nicht möglich, da beispielsweise eine Adresse in mehreren Orten vorkommen kann, wird die Zelle nicht geocodiert und der Text wird gelb hinterlegt. In solchen Fällen hat man die Möglichkeit den zugehörigen Ort manuell mit Hilfe einer Karte zu wählen.

\begin{figure}[!ht]
	\centering
	\includegraphics[width=\textwidth]{images/einfuehrung/geocoding_failed}
	\caption{Geocodierung für Ort Springfield fehlgeschlagen}
	\label{geocoding_failed}
\end{figure}

Die geocodierten Standorte werden als Metadaten in der Tabelle hinterlegt. Leider sind diese Daten über das SQL \gls{API} nicht selektierbar. Möchte man die erhaltenen Daten auf der Karte positionieren, so muss man für jede Zeile die \gls{Geocodierung} manuell vornehmen, was sich negativ auf die Ladezeit der Karte auswirkt.

\subsection{Geocoding-Dienste}
Es gibt verschiedene Dienste, welche eine solche \gls{Geocodierung} von Standortdaten anbieten. Die meisten davon haben aber eine Begrenzung der möglichen Anfragen pro Tag.

\begin{table}[H]
\centering
\begin{tabular}{|p{0.4\threecelltabwidth}|p{0.14\threecelltabwidth}|p{0.46\threecelltabwidth}|}
\hline 
\textbf{Anbieter} & \textbf{Anfragen pro Tag} & \textbf{URL} \\ 
\hline 
Google Maps Geocoding \gls{API} & 2500 & \url{https://developers.google.com/maps/documentation/geocoding/?hl=de} \\ 
\hline 
Yahoo! PlaceFinder \gls{API} & 50000 & \url{http://developer.yahoo.com/geo/placefinder/} \\ 
\hline 
MapQuest Geocoding \gls{API} & keine Begrenzung & \url{http://developer.mapquest.com/web/products/dev-services/geocoding-ws} \\ 
\hline 
\end{tabular}
\caption{Geocoding Limitierungen}
\end{table} 

\subsection{Geocoding von neuen Datensätzen nur manuell möglich}
\label{geocodierung-bug}
Ein grosses Problem stellt sich darin, dass neue Datensätze, welche über das SQL \gls{API} eingefügt wurden, nicht automatisch geocodiert werden. Dies führt dazu, dass diese Daten von Abfragen, welche eine ortsbezogene Einschränkung beinhalten (siehe Abschnitt \ref{sqlapi-spatialqueries}), nicht zurückgeliefert werden.

Eine \gls{Geocodierung} dieser Daten kann nur manuell über das Fusion Tables Web-GUI gestartet werden.


% Google Maps API - FusionTablesLayer
\section{Google Maps API - FusionTablesLayer}
\label{gmap-api-fusiontableslayer}
Google bietet von Haus aus bereits eine Fusion Table-Integration im Google Maps API V3 an. Damit ist es möglich Fusion Tables als eigenständige Layer direkt auf der Karte darzustellen.
Die Möglichkeiten dieser Layer sind noch stark eingeschränkt aber die grundlegenden Funktionalitäten für das Arbeiten mit Geodaten sind bereits vorhanden.

So ist es möglich Abfragen mit \inlinecode{WHERE}-Conditions einzuschränken oder die Stile des Layers selbst zu bestimmen.

\subsection{Karten-Stile}
\label{fusiontableslayer-styles}
Die FusionTablesLayer bieten ein abfragebasiertes Styling der Ebene an. Damit ist es möglich Flächen oder Linien farblich hervorzuheben oder andere Icons für Markierung zu verwenden. Zu jedem Stil kann man eine Einschränkung festlegen, welche bestimmt, ob dieser für den aktuellen Datensatz angewendet wird oder nicht. Die Einschränkung entspricht grundsätzlich einer \inlinecode{WHERE}-Condition in der Abfrage.

\emph{Hinweis: Passen mehrere Einschränkungen auf ein Datensatz, erhält dieser den Stil, welcher als letztes definiert wurde. Dies kann zu ungewollten Resultaten führen.}

\subsubsection{Beispiel eines Stils:}
\lstset{language=JavaScript}
\begin{lstlisting}[caption=Beispiel eines FusionTablesLayer-Stylings, label=fusiontableslayers-styles-example]
styles: [{
	polygonOptions: {
		fillColor: "#00FF00", // gruen
		fillOpacity: 0.3
	}
}, {
	where: "birds > 300",
	polygonOptions: {
		fillColor: "#0000FF" // blau
	}
}]
\end{lstlisting}

In diesem Beispiel\footnote{Quelle: \url{https://developers.google.com/maps/documentation/javascript/layers?hl=de-DE\#fusion_table_styles} (Stand: 21.05.2012)} werden alle Polygone der Tabelle, welche in der Spalte \emph{birds} eine Zahl grösser als 300 eingetragen haben, \emph{blau} eingefärbt. Die restlichen Polygone erhalten eine \emph{grüne} Färbung mit einer Deckkraft von 30\%.

\subsubsection{Einschränkungen}
\label{fusiontableslayer-styles-restrictions}
Das Google Maps API hat momentan folgende Einschränkungen bezüglich den Ebenenstilen definiert.

\begin{longtable}{|l|l|}
\hline 
\textbf{Beschreibung} & \textbf{Limitation} \\ 
\hline 
Anzahl Ebenen pro Karte & 5 \\ 
\hline 
Anzahl Ebenen auf denen Stile definiert sein dürfen & 1 \\ 
\hline 
Anzahl Stile auf Ebene & 5 \\ 
\hline 
\caption{Limitationen der Stile auf Fusion Table-Ebenen}
\label{fusiontableslayer-stlyes-limitations}
\end{longtable} 

\subsection{Heatmaps}
Ein weiteres Feature der Fusion Table-Ebenen ist die Möglichkeit die Daten der Tabelle direkt als Heatmap darzustellen. Dabei werden die Daten automatisch nach der Häufigkeit der Vorkommnisse an einem Ort anders eingefärbt. Der verwendete Farbverlauf geht dabei von \emph{grün} (für wenig Daten) bis \emph{rot} (für viele Daten).

\begin{figure}[!ht]
	\centering
	\includegraphics{images/einfuehrung/gmap_fusiontableslayer_heatmap}
	\caption{Daten als Heatmap mit FusionTablesLayer}
	\label{fusiontableslayer-heatmap}
\end{figure}

Leider sind momentan die Konfigurationsmöglichkeiten der Heatmap auf ein Minimum beschränkt. Eine Legende lässt sich beispielsweise nicht anzeigen. So kann man nicht genau sagen, welche Werte nun hinter den verschiedenen Farben stecken. Zudem lassen sich diese auch nicht verändern.

\subsection{Performance}
Der grösste Vorteil der Fusion Table-Ebenen steckt aber nicht zwingend in den sichtbaren Features. Man findet ihn wohl eher darin, dass die Geocodierungen der Standort-Daten direkt aus der Tabelle gelesen werden und nicht manuell vom Client abgefragt werden müssen. Dadurch kann die Ebene komplett auf den Servern von Google aufbereitet werden. Der Client muss die erhaltenen Daten lediglich noch darstellen. Der Vorteil davon wird durch das Diagramm \ref{fusiontableslayer-compare_markers}\footnote{Quelle: \url{http://www.google.com/events/io/2011/sessions/managing-and-visualizing-your-geospatial-data-with-fusion-tables.html} (Stand: 17.05.2012)} schnell ersichtlich.

\begin{figure}[!ht]
	\centering
	\includegraphics[width=0.9\textwidth]{images/einfuehrung/gmap_fusiontableslayer_vs_markers}
	\caption{FusionTablesLayer verglichen mit Markers}
	\label{fusiontableslayer-compare_markers}
\end{figure}

Die Zeit für das Rendering der Karte bleibt demnach bei der Verwendung von Fusion Table-Ebenen konstant und somit unabhängig von der Anzahl Markierungen, welche gesetzt werden müssen. Der Rechenaufwand, der für das Erstellen der JavaScript Marker-Objekte verwendet werden müsste, wird direkt von den Google Servern übernommen und das Resultat als Bild zum Client gesendet. Daraus resultiert die konstante Zeit, welche für die Anfrage zum Server und für das Senden der Antwort zum Client verwendet wird.

\subsection{Einschränkungen}
\subsubsection{Freigabe der Tabelle}
Ein grosser Nachteil der Fusion Table-Ebenen besteht darin, dass die verwendeten Fusion Tables als  \emph{öffentlich} markiert sein müssen, um diese auf einer Karte darzustellen. Sprich jeder kann die Tabellen anzeigen oder auslesen. Es ist also nicht möglich eine Tabelle mit sensiblen Daten als Fusion Table-Ebene darzustellen.

Von Google wird zur Lösung dieses Problems aber folgendes Vorgehen vorgeschlagen: Man kann für Tabellen mit sensiblen Inhalten eine View erstellen, welche lediglich die öffentlichen Spalten und Zeilen selektiert. Diese View könnte man dann als \emph{öffentlich} markieren und in einer Fusion Table-Ebene verwenden.

Eine Ausnahme dieser Regelung bilden dabei die Maps API Premier Kunden. Diese haben die Möglichkeit eine Fusion Table als \emph{Protectet Map Layer} freizugeben, wodurch sich diese nur in einer definierten Applikation als Ebene einbinden lässt. Die Tabelle bleibt dabei komplett privat und kann nicht ausgelesen werden.

\subsubsection{Event-Handling auf Fusion Table-Ebenen}
Bislang ist es erst möglich den Klick-Event auf einer Fusion Table-Ebene zu behandeln. Dieser liefert standardmässig das HTML-Template des InfoWindows mit, welches mit dem Klick angezeigt wird. Zudem erhält man die Daten der "`angeklickten Zeile"' beziehungsweise die zugehörigen Daten des angeklickten Objekts auf der Karte. Man hat die Möglichkeit dieses Template anzupassen bevor das InfoWindow angezeigt wird.

Weitere Events können auf einer Fusion Table-Ebene aber noch nicht behandelt werden. Es ist also nicht möglich beispielsweise ein Objekt auf der Karte bei einem MouseOver-Event anders einzufärben oder ähnliches.

Diese fehlenden Möglichkeiten wurden schon oft von anderen FusionTables-Benutzern bei den Google-Entwicklern angefordert, welche dies ebenfalls als ein wichtiges Feature sehen, dass noch implementiert werden muss.

Als Übergangslösung findet sich bereits eine Custom Library mit dem Namen \emph{Fusion Tips}\footnote{\url{http://gmaps-utility-gis.googlecode.com/svn/trunk/fusiontips/docs/examples.html}}. Diese legt über die Fusion Table-Ebene eine weitere transparente Ebene. Auf dieser Ebene kann man alle Events, welche das Google Maps API anbietet behandeln (somit auch den MouseOver-Event). Sobald die Maus über die Position eines Elementes auf der Fusion Table fährt, sendet der Event einen Request über das SQL API an die Fusion Table und holt sich die Informationen zu der entsprechenden Zeile. So ist es möglich ein neues Element mit der gleichen Form aber beispielsweise einer anderen Farbe darüber zu legen.

Für den Benutzer entsteht so der Effekt als würde sich die Farbe des gehoverten Elementes ändern.

Die Libary wird in folgendem Blog-Artikel sehr gut erklärt: \url{http://csessig.wordpress.com/2012/02/07/multiple-layers-and-rollover-effects-for-fusion-table-maps/}

\subsubsection{Weitere Einschränkungen}
Zusätzlich sind die allgemeinen Limitationen der Google Fusion Tables (siehe Tabelle \ref{gft-limitations}) zu beachten.


% Google Fusion Table Javascript Library (GftLib)
\section{Google Fusion Table Javascript Library (GftLib)}
\label{gftlib-js}
Die Google Fusion Table Javascript Library vereinfacht die Kommunikation mit dem Google Fusion Table SQL API. Sie hilft dabei SQL-Queries zu erstellen und per AJAX an das API zu versenden.

Zur Erstellung der AJAX-Requests werden die \$.get()- und \$.post()-Helpermethoden der jQuery Library in der Version 1.7.1 (Minified) verwendet.

\todo[inline]{GftLib Abschnitt überarbeiten}

\subsection{Abhängikeiten}
\begin{longtable}{|l|l|p{11.5cm}|}
\hline 
\textbf{Library} & \textbf{Version} & \textbf{Verwendung} \\ 
\hline 
jQuery & 1.7.1-min & AJAX-Requests und weitere Helper-Funktionen \\ 
\hline 
\end{longtable} 

\subsection{Methoden}
\begin{longtable}{|l|p{4.5cm}|p{5cm}|}
\hline 
\textbf{Methode} & \textbf{Beschreibung} & \textbf{Parameter} \\ 
\hline 
execSql(callback, query) & Führt einen SQL-Befehl & callback (Funktion): Callback-Methode welche nach Beendigung der Methode aufgerufen wird. query (String): SQL-Query \\ 
\hline 
execSelect(callback, options) & Führt einen SQL-Abfrage aus & callback (Funktion): Callback-Methode welche nach Beendigung der Methode aufgerufen wird. query (String): SQL-Query \\ 
\hline 
convertToObject(gftData) & Konvertiert das Resultat einer Abfrage in sprechende Objekte & • \\ 
\hline 
\end{longtable} 


% Authentifizierung mit OAuth
\section{Authentifizierung mit OAuth}
\label{oauth}
\todo[inline]{OAuth Abschnitt schreiben}

% Infrastruktur
\chapter{Infrastruktur}
\label{infrastruktur}
\section{Server in der Cloud mit Amazon EC2}
Die Prämisse war von Anfang an klar: möglichst alle Dienste welche wir für diese Studienarbeit brauchen, sollen aus der \emph{\gls{Cloud}} kommen. Da liegt es natürlich Nahe, dies auch auf die Infrastruktur anzuwenden. Dank dem Amazon Free Usage Tier\footnote{\url{http://aws.amazon.com/free/}} konnten wir kostenlos Server bei Amazon EC2 aufsetzen.

Gemäss dem \gls{Cloud}-Ansatz, sind Ressourcen relativ kostengünstig, weshalb wir für jede Aufgabe einen separaten Server verwendet haben. So haben wir auch die Kopplung reduziert und können einzelne Dienste sehr einfach austauschen.

Insgesamt haben wir 3 Server verwendet:
\begin{itemize}
\item Build- und Deploymentserver: Build und Hosting der Sourcen
\item PhantomJS-Server: Ausführung der JavaScript-Tests 
\item Redmine-Server: Hosting der Projektmanagement-Software
\end{itemize}

Den Build- und Deploymentserver und den PhantomJS-Server haben wir selbst aufgesetzt. Den Redmine-Server mussten wir lediglich konfigurieren. Dies weil wir dafür auf einem bestehenden Image aufsetzen konnten, auf welchem Redmine schon installiert war. Das Open-Source Projekt BitNami\footnote{\url{http://bitnami.org/}} erstellt für zahlreiche Projekte solche Images. Dadurch entfällt ein Grossteil des Installationsaufwandes.

\section{Unit-Testing mit PhantomJS und QUnit}
Da wir hauptsächlich JavaScript als Programmiersprache verwendet haben, stellte sich bald einmal die Frage, wie sich unser Code sinnvoll testen lässt. Da wir bereits aus einem anderen Projekt sehr gute Erfahrungen mit QUnit\footnote{QUnit ist ein Unit Testing Framework von jQuery: \url{http://docs.jquery.com/QUnit}} gesammelt haben, wollten wir darauf aufbauen und haben damit unseren Code geprüft.

Da es sich auch bei unserem Testcode um JavaScript handelt, sind wir immer auf einen Browser angewiesen, welcher entsprechend die Tests ausführt. Für einen automatisierten Build ist dies unzureichend, da dieser nicht direkt die Tests ausführen kann.

Nach einigem Suchen sind wir schliesslich auf das Projekt PhantomJS\footnote{\url{http://phantomjs.org/}} aufmerksam geworden. Dabei handelt es sich um einen \gls{HeadlessBrowser}, welcher ohne grafische Darstellung auskommt. Damit konnten wir dann die URL unserer Tests\footnote{\url{http://gft.rdmr.ch/test/js/}} ansteuern und das Resultat in Jenkins auswerten\footnote{\url{http://phantomjs.rdmr.ch/gftprototype.php}}.

Es hat sich dann noch als Schwierigkeit ergeben, dass wir den Output von QUnit in eine für Jenkins verständliche Form bringen mussten. Dazu haben wir den Testrunner \inlinecode{run\_qunit.js} erweitert und mit zusätzlichen Ausgabemöglichkeiten ausgestattet. Alle nötigen Schritte haben wir schliesslich als Anleitung in einem Blogpost\footnote{\url{http://www.readmore.ch/post/18940470535}} veröffentlicht. Wir haben auch schon Feedback\footnote{\url{https://github.com/odi86/GFTPrototype/issues/1}} dazu erhalten und konnten so unsere Lösung weiter verbessern.

\section{Continuous Integration mit dem Jenkins Build-Server}
\label{build-server}
Bei Jenkins CI\footnote{\url{http://jenkins-ci.org/}} handelt es sich um ein Open Source Projekt, welches Unterstützung bietet für die Projektautomatisierung. Es unterstützt zahlreiche Repositories und Build-Mechanismen. Wir haben auch einige Plugins aktiviert, u.a. die GitHub- und Redmine-Integration oder der Logfile-Parser.

\subsection{Logfile-Analyse mit dem Log Parser Plugin}
Das Log Parser Plugin\footnote{\url{http://wiki.hudson-ci.org/display/HUDSON/Log+Parser+Plugin}} ermöglicht es auf einfache Art und Weise Probleme mit dem Build festzustellen. Für den Build kann es zum Teil sehr schwierig sein, den Status des Builds zu bestimmen. Der Build kann erkennen ob ein aufgerufener Befehl fehlgeschlagen hat oder ob die ausgeführten Tests erfolgreich durchlaufen. Wenn jedoch ein aufgerufenes Programm eine Warnung oder einen Fehler anzeigt, kann der Build das nicht selbst erkennen. Hier kommt der Logparser ins Spiel. Man kann Muster definieren für Dinge die auf eine Warnung oder einen Fehler hindeuten. Diese Logik erlaubt es dem Build viel detailliertere Aussagen über die laufende Installation zu machen.

\begin{figure}[!ht]
	\centering
	\includegraphics[width=\textwidth]{images/infrastruktur/jenkins-parsed-output}
	\caption{Ansicht des Logfiles mit dem Log Parser Plugin}
	\label{jenkins-parsed-output}
\end{figure}

\subsection{Trigger (Auslöser)}
Wir haben unseren Buildserver so getriggert, dass dieser ausgelöst wird, wenn eine Änderung auf GitHub gepusht wurde. Dazu haben wir das GitHub Plugin\footnote{\url{https://wiki.jenkins-ci.org/display/JENKINS/Github+Plugin}} auf Jenkins installiert, welches sich einfach im Backend installieren lässt. Im GitHub-Repository mussten wir dann noch sogenannte Web Hooks{\footnote{\url{http://help.github.com/post-receive-hooks/}} einrichten, so dass GitHub jeweils Jenkins benachrichtigt, sobald Änderungen eintreffen.

\subsection{Build Steuerung mit Ant}
Zur Build-Steuerung verwenden wir Ant\footnote{\url{http://ant.apache.org/}}. Ant ist vor allem aus der Java-Welt bekannt, lässt sich aber auch gut in andere Umgebungen einbinden. Der Build wir in einer XML-Datei spezifiziert. Einzelne Schritte darin werden \emph{\gls{AntTarget}} genannt und diese sollten grundsätzlich von einander unabhängig sein. Dies ermöglicht es auch nur Teilschritte eines Builds auszuführen. Es lassen sich aber auch Abhängigkeiten zwischen Targets definieren, welche dann der Reihe nach abgearbeitet werden.

Hier zum Beispiel unser \inlinecode{build}-Target, welches alle anderen Targets als Abhängigkeit hat:
\lstset{language=Ant}
\begin{lstlisting}
<target name="build" depends="test,documentation,generate_css,deploy,test-js">
	<echo message="BUILD FINISHED!"/>
</target>
\end{lstlisting}

\subsection{Testausführung}
Zur Darstellung der Testresultate bauen wir auf dem JUnit XML-Interpreter auf, welcher in Jenkins bereits integriert ist. Dazu ist es lediglich notwendig alle Testresultate in das JUnit XML-Format zu bringen, die ganzen Auswertungen, Grafiken etc. bekommt man dann quasi "`gratis"' dazu.

\begin{figure}[!ht]
	\centering
	\includegraphics[width=0.8\textwidth]{images/infrastruktur/jenkins-test-results}
	\caption{Diagramm der Testresultate im Verlaufe der Zeit}
	\label{jenkins-test-results}
\end{figure}


% Use Case 1: WorldData
\chapter{Use Case 1: WorldData Explorer}

\section{Einführung}
Im ersten Use Case geht es hauptsächlich um die Anzeige grosser Datenmengen auf der Karte. Dazu importieren wir bestehende Datenbestände in die Google Fusion Tables und visualisieren diese mittels Google Maps API auf der Karte.

\subsection{Ziel}
Es sollen verschiedene historische Länderdaten auf einer Weltkarte angezeigt werden. Die Daten sind pro Jahr und Thema unterteilt. Über eine Zeitachse soll es möglich sein die Daten der verschiedenen Jahre zu selektieren. Eine solche Darstellung kann beispielsweise dabei helfen Zusammenhänge zwischen verschiedenen Themenbereichen zu finden.

\subsection{Vorgehen}
Um die Daten pro Land zu visualisieren, werden zuerst die Landesgrenzen als Geometrie-Datensätze in eine separate Fusion Tabelle importiert. In eine andere Tabelle werden dann die Daten importiert unterteilt nach Land und Jahr. Diese beiden Tabellen werden per Merge-Funktion zu einer Tabelle zusammengefasst, welche dann mittels FusionTablesLayer auf der Karte dargestellt werden kann.

\subsubsection{ERD}
\includegraphics[scale=0.8]{images/usecase1-worlddata-erd.png}

\subsection{Datenquellen}
Als Datenquelle wurde der Daten Katalog der Weltbank (\url{http://data.worldbank.org/}) verwendet. Darin finden sich länderspezifische Daten aufgeteilt in über 7000 Themenbereiche.

Die Landesgrenzen wurden als \gls{KML}-Datei von einer inoffiziellen Google Earth Library-Webseite (\url{http://www.gelib.com/world-borders.htm}) bezogen. 

\section{Anforderungsspezifikation}

\section{Analyse}

\section{Design}

\section{Implementation}

\section{Test}

\section{Resultate}

\section{Weiterentwicklung}

\section{Benutzerdokumentation}
\subsection{Importieren der Daten in Google Fusion Tables}
\subsubsection{Landesgrenzen}
Die Landesgrenzen liegen als KML-Datei vor. Diese beinhaltet alle Länder mit ihren Grenzen definiert als Polygone.

\includegraphics{images/usecase1-worlddata/worlddata-worldborders_kml.png}

\begin{enumerate}
\item Google Docs öffnen (\url{https://docs.google.com/}) und sich mit seinem Google Account einloggen
\item Erstellen > Tabelle (Beta) \\ \includegraphics{images/usecase1-worlddata/worlddata-worldborders_import1.png}
\item Es öffnet sich die neue Fusion Table mit dem Dialog zum Importieren von bestehenden Daten
\item Im Tab \emph{From this computer} die lokal gespeicherte Datei auswählen > Next
\item Die Daten werden automatisch in passende Spalten eingeteilt
\item Abschliessend muss der Tabelle noch einen Namen gegeben werden
\item Mit \emph{Finish} werden die Daten dann importiert
\end{enumerate}

Die Tabelle sollte nun folgendermassen aussehen:

\includegraphics[scale=0.65]{images/usecase1-worlddata/worlddata-worldborders_import_done.png}

\subsubsection{Daten}
Die Daten liegen als CSV-Datei vor. Diese beinhaltet folgende Spalten:
\begin{itemize}
\item Country Name
\item Country Code
\item Indicator Name
\item Indicator Code
\item 1960
\item 1961
\item ...
\end{itemize}

\includegraphics{images/usecase1-worlddata/worlddata-data_csv.png}

Das Vorgehen für den Import der Daten ist dasselbe wie bei den Landesgrenzen.

Die Tabelle sollte nun folgendermassen aussehen:

\includegraphics[scale=0.5]{images/usecase1-worlddata/worlddata-data_import_done.png}

\subsubsection{Tabellen mergen}
Sind beide Fusion Tables erstellt müssen diese zusammengefügt werden, um sie schlussendlich als einen einzelnen Layer auf der Karte darzustellen. Dazu bietet Google Fusion Tables die \emph{Merge}-Funktion an. 

\begin{enumerate}
\item Eine der beiden Tabellen öffnen
\item Im Menü \emph{Merge} auswählen \\ \includegraphics{images/usecase1-worlddata/worlddata-merge1.png}
\item Es öffnet sich ein Popup, welches durch den Vorgang führt
\item In der 1. Spalte wählt man die Spalte aus, über welche die beiden Tabellen verbunden werden sollen. In unserem Fall ist dies die Spalte mit den Namen der Länder.
\item In der 2. Spalte muss man zuerst die andere Tabelle auswählen und dann ebenfalls die Spalte, in welcher die Namen der Länder gespeichert sind.
\item Schlussendlich muss der neuen Tabelle ein Namen gegeben werden
\item Mit einem Klick auf \emph{Merge tables} wird die neue Tabelle mit den zusammengefügten Daten erstellt. \\ \includegraphics[scale=0.8]{images/usecase1-worlddata/worlddata-merge2.png}
\end{enumerate}

Die zusammengeführte Tabelle sollte nun folgendermassen aussehen:

\includegraphics[scale=0.4]{images/usecase1-worlddata/worlddata-merge_done.png}

\subsubsection{Merge-Tabelle für die Verwendung mit FusionTablesLayer vorbereiten}
Um die Tabelle nun als FusionTablesLayer verwenden zu können, muss diese als \emph{öffentlich} markiert werden. Dazu klickt man bei der geöffneten Tabelle auf den \emph{Share}-Button in der linken oberen Ecke.

\includegraphics{images/usecase1-worlddata/worlddata-prepare_fusiontableslayer1.png}

Im Dialogfenster wählt man unter \emph{Visibility options} den Eintrag \emph{Public} aus.

\includegraphics[scale=0.8]{images/usecase1-worlddata/worlddata-prepare_fusiontableslayer2.png}

Als letzten Schritt muss man sich noch die eindeutige ID der Tabelle merken. Dazu wählt man im Menü  \emph{File > About} und kopiert sich die angezeigte \emph{Numeric ID} im geöffneten Dialog.

\includegraphics{images/usecase1-worlddata/worlddata-prepare_fusiontableslayer3.png}

\includegraphics{images/usecase1-worlddata/worlddata-prepare_fusiontableslayer4.png}

\subsection{Konfiguration der Applikation}


% Use Case 2: FixMyStreet
\chapter{Use Case 2: FixMyStreet}

\section{Einführung}
FixMyStreet ist ein Konzept, welches bereits in verschiedenen Städten bzw. Ländern umgesetzt wurde. Beispiele dafür sind Deutschland welches dazu die Webseite  \url{http://de.seeclickfix.com/} anbietet oder England mit der Webseite \url{http://www.fixmystreet.com/}. Beide Beispiele bieten neben der Webseite auch eine native App für iOS und Android an.

Die Idee hinter dem Konzept ist so einfach wie auch genial. Man ermöglicht dem Bürger per Webseite oder App entdeckte Defekte in seiner Umgebung (defekte Strassenlampen, Schlaglöcher, usw.) direkt der Stadt oder Gemeinde zu melden. Diese kann dann diese Meldungen überprüfen und wenn nötig beheben. So können teure Kontrollfahrten auf ein Minimum reduziert werden.

\subsection{Ziel}
Das Ziel dieses UseCases war die Erstellung einer WebApp, welche genau dieses Konzept umsetzt. Die Benutzer sollen die Möglichkeit haben Defekte in ihrer Umgebung dem zuständigen Amt zu melden.

Google Fusion Table soll dazu als Datenbank verwendet werden, in der die Defekte abgelegt werden. Natürlich sollen auch einige GIS-Features der Fusion Table verwendet werden, um beispielsweise nur die Defekte im aktuell sichtbaren Bereich der Karte zu laden.

\section{Anforderungsspezifikation}
\subsection{Use Cases}
\begin{figure}[!h]
	\centering
	\includegraphics[scale=0.5]{images/usecase2-fixmystreet/fixmystreet-usecasediagram.png}
	\caption{FixMyStreet UseCase Diagramm}
	\label{fixmystreet-usecasediagram}
\end{figure}

\subsubsection{Use Case 1a: Defekt melden}
\paragraph{Primary Actor}
\begin{itemize}
\item Bürger
\end{itemize}

\paragraph{Stakeholders and Interests}
\begin{itemize}
\item Bürger: Möchte einen entdeckten Defekt melden
\end{itemize}

\paragraph{Preconditions}
\begin{itemize}
\item WebApp ist gestartet
\end{itemize}

\paragraph{Success Guarantee (Postconditions)}
\begin{itemize}
\item Neuer Defekt ist in Datenbank gespeichert
\item Defekt ist in der Liste und auf der Karte ersichtlich
\item Melde-Maske wurde in den Ursprungszustand zurückgesetzt (kein Defekttyp ausgewählt, Markierung wieder auf die aktuelle Position verschoben)
\end{itemize}

\paragraph{Main Success Scenario}
\begin{enumerate}
\item Bürger wählt Defekttypen aus
\item Bürger verschiebt Markierung auf Karte zur Position des Defekts
\item Bürger sendet den Defekt ab
\item Bürger bestätigt die Kontrollfrage, ob der defekt tatsächlich gesendet werden soll
\end{enumerate}

\paragraph{Alternative Flows}
2a. Markierung muss nicht zwangsläufig verschoben werden. Sie zeigt beim Starten der WebApp auf die aktuelle Position. Falls dies nicht zugelassen wird zeigt sie auf einen vorkonfigurierten Ort.

\paragraph{Special Requirements}
-

\paragraph{Frequency of Occurrence}
Tritt sehr häufig auf, da eine beliebige Anzahl von Bürgern Defekte melden kann.

\subsubsection{Use Case 1b: Defekt löschen}
\paragraph{Primary Actor}
\begin{itemize}
\item Bürger
\end{itemize}

\paragraph{Stakeholders and Interests}
\begin{itemize}
\item Bürger: Möchte einen bereits gemeldeten Defekt löschen
\end{itemize}

\paragraph{Preconditions}
\begin{itemize}
\item WebApp ist gestartet
\item Die Listenansicht wurde geöffnet 
\item Es wurde bereits ein Defekt gemeldet
\end{itemize}

\paragraph{Success Guarantee (Postconditions)}
\begin{itemize}
\item Der Defekt wurde aus der Datenbank gelöscht
\item Defekt ist nicht mehr auf der Liste und auf der Karte ersichtlich
\end{itemize}

\paragraph{Main Success Scenario}
\begin{enumerate}
\item Bürger markiert den zu löschenden Defekt in der Liste
\item Bürger wählt den Befehl \emph{Löschen}
\end{enumerate}

\paragraph{Alternative Flows}
2a. Falls der Status des Defektes bereits von einem Gemeindearbeiter geändert wurde, ist es für den Bürger nicht mehr möglich den Defekt zu löschen.

\paragraph{Special Requirements}
-

\paragraph{Frequency of Occurrence}
Tritt sehr häuft auf, da es für jeden Bürger, welcher bereits einen Defekt gemeldet hat, möglich ist seine eigenen Defekte wieder zu löschen.

\section{Analyse}

\section{Design}

\subsection{ERD}

\subsection{Klassendiagramm}

\section{Implementation}
\subsection{Systemanforderungen:}
Die WebApp basiert auf dem Sencha Touch 2 Framework. Dieses bietet eine Basis zur Erstellung von mobilen WebApps mit JavaScript. Das Sencha Touch 2 Framework unterstützt alle WebKit-fähigen Browser:

\paragraph{Desktop:}
\begin{itemize}
\item Chrome
\item Opera
\item Safari
\end{itemize}

\paragraph{Mobile:}
\begin{itemize}
\item iOS
\item Android
\item Blackberry
\end{itemize}

\subsection{Abhängigkeiten}
\begin{tabular}{|l|l|}
\hline 
Library & Version \\ 
\hline 
Sencha Touch 2 & 2.0.1 \\ 
\hline 
jQuery & 1.7.1 \\ 
\hline 
Google Maps API & V3 \\ 
\hline 
gftlib-js & 1.0 \\ 
\hline 
\end{tabular} 

\section{Implementation}
\subsection{Quellcode-Struktur}
\begin{tabular}{|l|l|}
\hline 
Datei & Beschreibung \\ 
\hline 
resources/images/ & Bilder der Applikation \\ 
\hline 
resources/styles/ & CSS-Styles der Applikation \\ 
\hline 
app/app.js & Einstiegspunkt der Applikation \\ 
\hline 
app/controller/List.js & Kontroller der \emph{List}-View \\ 
\hline 
app/controller/Map.js & Basis-Kontroller der \emph{Report}-View und der \emph{Map}-View \\ 
\hline
app/controller/ProblemMap.js & Kontroller der \emph{Map}-View \\ 
\hline
app/controller/ReportMap.js & Kontroller der \emph{Report}-View \\ 
\hline
\end{tabular} 

\section{Test}

\section{Resultate}

\section{Weiterentwicklung}

\section{Benutzerdokumentation}


% Converter-Build
\chapter{Converter-Build}
\label{converter-build}

\section{Idee}
Der Converter-Build war zu Beginn ein sehr einfaches Hilfsmittel um Testdaten in Google Fusion Tables zu laden. Das Problem war, dass GFT von sich aus nur den Import von Google Spreadsheet, \gls{CSV}- und \gls{KML}-Dateien zu lässt, andere Formate sind nicht unterstützt. Viele frei verfügbare \gls{GIS}-Daten sind jedoch in anderen Formaten verfügbar, so dass eine Konvertierung zwangläufig notwendig ist. Zuerst haben wir jeweils den GeoConverter der HSR für diesen Zweck verwendet. Der Prozess ist allerdings etwas mühsam, da die konvenrtierte Datei dann jeweils noch manuell in GFT importiert werden muss.

Der Converter-Build ermöglicht es einfach und schnell beliebige Dateien zu konvertieren oder in GFT zu importieren.

\section{Implementation}
\subsection{Technischer Hintergrund}
Der Converter-Build basiert auf der Open Source Bibliothek GDAL/OGR.\footnote{\url{http://www.gdal.org/ogr}}.

Technisch gesehen ist der Converter-Build ein einfaches Ant-Skript welches das Kommandozeilen-Tool ogr2ogr\footnote{\url{http://www.gdal.org/ogr2ogr.html}} abstrahiert. Dieses Tool wird auf dem Server ausgeführt mit den entsprechend eingegebenen Parametern. 

\subsection{Features}
Da der Build auf ogr2ogr basiert, sind die Limitation durch dieses Tool gegeben. Es wird eine Vielzahl an \gls{GIS}-Formaten unterstützt\footnote{\url{http://www.gdal.org/ogr/ogr_formats.html}}. Speziell zu erwähnen ist dabei die Unterstützung für Google Fusion Tables, da wir damit die Möglichkeit bekommen haben, Daten direkt in GFT zu importieren\footnote{\url{http://www.gdal.org/ogr/drv_gft.html}}.

Für die anderen Formate wird jeweils das konvertierte File wieder zum Download angeboten.

\begin{figure}[!ht]
	\centering
	\includegraphics[width=\textwidth]{images/converter-build/converter-build-done}
	\caption{Wenn der Build abgeschlossen ist, lässt sich die konvertierte Datei herunterladen}
	\label{converter-build-import}
\end{figure}

Der Converter-Build nimmt sich die Input-Datei wahlweise von einem Upload oder von einer angegebenen URL. Weiter lässt sich das gewünschte Dateiformat und eine allfällige Koordinatenkonvertierung einstellen. Falls als Ausgabeformat \emph{GFT} gewählt wurde, muss man noch seine Angaben zu seinem Google-Konto angeben.

\begin{figure}[!ht]
	\centering
	\includegraphics[width=\textwidth]{images/converter-build/converter-build-import}
	\caption{Optionen für den Converter-Build}
	\label{converter-build-import}
\end{figure}

\section{Installation}
Die Voraussetzungen für den Converter-Build sind zunächst eine Jenkins-Instanz, auf welcher das Plugin \emph{Parametetrized Build} installiert ist\footnote{\url{https://wiki.jenkins-ci.org/display/JENKINS/Parameterized+Build}}. Damit lassen sich an den Build via Web GUI Parameter übergeben. Grundsätzlich können damit generische Build-Jobs erstellt werden und dann je nachdem verschieden benutzt werden. 

Das Herzstück des Builds ist die GDAL/OGR Bibliothek welche sich aus dem Quellcode kompilieren lässt. Wichtig ist dabei, dass die Option \inlinecode{-{}-with-curl} angegeben wird, da ansonsten der GFT-Import nicht funktioniert.

\begin{lstlisting}
$ ./configure --with-curl
$ make
# make install
\end{lstlisting}

Für die Koordinaten-Konvertierung muss zusätzlich noch die Bibliothek Proj.4\footnote{\url{http://trac.osgeo.org/proj/}} installiert werden.


% Projektmanagement
\part{Projektmanagement}
\chapter{Projektmanagement}

\section{Allgemeines}

\section{Projektmanagement}
Wir verwendeten die agile Projektmethologie Scrum\footnote{\url{http://www.scrum.org}} und arbeiteten dabei während 12 Wochen (4 Sprints à 3 Wochen pro Sprint).
Für diese Studienarbeit werden 8 ECTS-Punkte vergeben, wobei 1 ECTS-Punkt 30 Stunden Arbeitsaufwand bedeuten.
Dies ergibt 240 Stunden pro Person, was etwas mehr als 2 Arbeitstagen pro Woche entspricht.

Als Ausgangslage haben wir 12 Storypoints pro Sprint angenommen, wobei 1 Storypoint ungefähr einem Arbeitstag entspricht. Bis zum Schluss hat sich dieser Wert nicht merklich verändert.

% Kickoff
\subsection{Kickoff}
\todo[inline]{Kickoff schreiben}

% Sprint 1
\subsection{Sprint 1}

Der erste Sprint stand ganz im Zeichen des Ausprobierens. Es ging uns darum mit dem Google Fusion Tables API bekannt zu werden. Daneben musste noch unsere Infrastruktur eingerichtet werden. Dazu zählt das Repository, unser Projektmanagement-Tool und der Build-Server.

Alle Informationen zum Sprint 1 sind auch in unserem Wiki zu finden:
\url{http://redmine.rdmr.ch/redmine/projects/gftprototype/wiki/Sprint_1}

\subsubsection{Hauptaufgaben / Fokussierung im Sprint}

\begin{itemize}
	\item Aufsetzen der Infrastruktur (Repository, Projektmanagement, Entwicklungsumgebung, Server)
	\item Einarbeitung in die Thematik
	\begin{itemize}
		\item GIS
		\item Google Fusion Tables (GFT)
		\item allenfalls weitere APIs
	\end{itemize}
	\item Erarbeitung eines ersten Roundtrips um Daten von und zu Google Fusion Tables zu schicken/empfangen
	\item Projektsetup
	\begin{itemize}
		\item GitHub\footnote{\url{http://www.github.com}} als Repository
		\item LaTex\footnote{\url{http://www.latex-project.org}} für Dokumentation
		\item Jenkins \footnote{\url{http://jenkins-ci.org}} für Continuous Integration (CI)
		\item Redmine \footnote{\url{http://www.redmine.org}} für Projektmanagement/Wiki/Bugtracker
		\item Scrum als Methodik
	\end{itemize}
\end{itemize}

\subsubsection{Ziele}
\begin{itemize}
	\item Google Fusion Table API kennen lernen, Potential abschätzen können
	\item Erster Prototyp mit GFT bauen (Roundtrip mit CRUD-Operationen)
\end{itemize}

\subsubsection{Abgabe / Deliverables}
Wir sind im ersten Sprint gut vorangekommen und konnten mit zahlreichen aufeiander aufbauenden Beispielen lernen wie das API funktioniert und welche Möglichkeiten es bietet: Abfragen erstellen, Zugriff via API, Zugriff via Google Maps Layer (FusionTablesLayer).

\begin{itemize}
	\item Infrastruktur aufgesetzt (Repository, Build-Server und Projekmanagement-Tool)
	\item Lauffähiger Prototyp mit Unit-Test für CRUD-Operationen
	\item Zahlreiche Beispiele um die Funktionsweise des APIs zu testen
\end{itemize}

\subsubsection{Probleme}
Es ist uns nicht gelungen den Roundtrip zu erstellen, d.h. Daten vom Benutzer in Fusion Tables zu speichern und dies dann wieder abzufragen. Wie sich herausgestellt hat, müssen schreibende Zugriffe authorisiert sein. Dazu empfiehlt Google OAuth zu benutzen. Diese Thematik war schlicht zu gross um im ersten Sprint anzuschauen. Die schreibenden Zugriffe sind denn auch Ziel für den 2. Sprint geworden. 

% Sprint 2
\subsection{Sprint 2}

Im zweiten Sprint gab es 2 Schwerpunkte: zum einen mussten wir uns langsam Gedanken machen, welche Use Cases wir mit den Google Fusion Tables abdecken wollen. Aus diesem Use Cases sollen dann unabhängige Applikationen entstehen, welche so dann das Potential des Produktes aufzeigen sollen. Zum anderen gab es noch ein wichtiges technischen Thema, nämlich die Schreiboperationen. Dazu waren einige Grundlagen von OAuth nötig, so dass wir dann die ganze Bandbreite der Schnittstelle nutzen konnten.

Als Nebenthema mussten wir uns noch um den Import von verschiedenen GIS Dateien in Fusion Tables kümmern. Zum einen ist dies ein sehr relevantes Thema um eine Migration zu ermöglichen, zum anderen sind viele Daten in beliebigen Formate verfügbar, welche wir natürlich gern als Testdaten nutzen möchten.

Alle Informationen zum Sprint 2 sind auch in unserem Wiki zu finden:
\url{http://redmine.rdmr.ch/redmine/projects/gftprototype/wiki/Sprint_2}

\subsubsection{Hauptaufgaben / Fokussierung im Sprint}
\begin{itemize}
	\item Use Cases erarbeiten
	\item GIS Daten in GFT importieren
	\item Informationen über das "Trusted Tester API" sammeln
	\item Schreiboperationen (INSERT/UPDATE/DELETE) auf Fusion Tables durchführen können (Authentifizierung mit OAuth notwendig)
\end{itemize}

\subsubsection{Ziele}
\begin{itemize}
	\item  Finden von WebGIS Use Cases (1 \emph{grosser} und 2 \emph{kleine} Use Cases)
	\item  Geo-Daten importieren und verknüpfen
	\begin{itemize}
		\item KML importieren
		\item Converter einsetzen, dann importieren
		\item Verschiedene Fusion Tables joinen/mergen
	\end{itemize}
	\item Trusted Tester API
	\begin{itemize}
		\item Zugriff erhalten
		\item API testen
	\end{itemize}
\end{itemize}

\subsubsection{Abgabe / Deliverables}
Wir sind im ersten Sprint gut vorangekommen und konnten mit zahlreichen aufeiander aufbauenden Beispielen lernen wie das API funktioniert und welche Möglichkeiten es bietet: Abfragen erstellen, Zugriff via API, Zugriff via Google Maps Layer (FusionTablesLayer).

\begin{itemize}
	\item Infrastruktur aufgesetzt (Repository, Build-Server und Projekmanagement-Tool)
	\item Lauffähiger Prototyp mit Unit-Test für CRUD-Operationen
	\item Zahlreiche Beispiele um die Funktionsweise des APIs zu testen
\end{itemize}

\subsubsection{Probleme}
Es is 

% Sprint 3
\subsection{Sprint 3}

Im dritten Sprint war das Hauptziel die Umsetzung des zweiten Use Cases \emph{FixMyStreet} als mobile WebApp. Dies beinhaltete unter anderem auch die Anbindung des Sencha Touch Frameworks an die Google Fusion Table.
Zusätzlich wollten wir noch einen Converter-Build erstellen, mit welchem es möglich ist verschiedenste GIS-Formate in eine Google Fusion Table zu importieren oder in andere Formate zu konvertieren.

Alle Informationen zum Sprint 3 sind auch in unserem Wiki zu finden:
\url{http://redmine.rdmr.ch/redmine/projects/gftprototype/wiki/Sprint_3}

\subsubsection{Hauptaufgaben / Fokussierung im Sprint}
\begin{itemize}
	\item Zweiter Use Case \emph{FixMyStreet} implementieren
	\item Converter-Build erstellen
\end{itemize}

\subsubsection{Ziele}
\begin{itemize}
	\item Lauffähige App für \emph{FixMyStreet} Use Case
	\begin{itemize}
		\item Eintragen von \emph{Defekten}
		\item Anzeigen der Defekte auf Karte
		\item Daten von GFT lesen / in GFT schreiben
	\end{itemize}
\end{itemize}

\subsubsection{Abgabe / Deliverables}
\begin{itemize}
	\item WebApp \emph{FixMyStreet} (erste Version)
	\item Converter-Build (GIS Formate konvertieren bzw. in GFT importieren)
\end{itemize}

Am Ende dieses Sprints hatten wir eine lauffähige Version der \emph{FixMyStreet}-WebApp (Kapitel \ref{fixmystreet}). Diese beinhaltete die Anbindung einer Google Fusion Table als Datenbank. Es sind aber noch einige kleine Bugs vorhanden, welche im nächsten Sprint korrigiert werden müssen.

\subsubsection{Probleme}
In der App sind noch einige Bugs in Bezug auf das Schreiben und Lesen der Google Fusion Table vorhanden. Es scheint als gäbe es Probleme mit der internen und externen ID der Datensätze. Zudem wurde bei verschiedenen Tests noch einige Usability-Probleme festgestellt, welche noch behoben werden müssen.\todo{weitere Probleme in Sprint 3 beschreiben}

% Sprint 4
\section{Sprint 4}

Im letzten Sprint ging es noch um das Finalisieren aller Arbeiten. Dies beinhaltete die Behebung der im Sprint 3 gefundenen Probleme in der \emph{FixMyStreet} \gls{WebApp}. Der grösste Teil der Zeit wurde aber für das Schreiben der Dokumentation eingeplant. 

Alle Informationen zum Sprint 4 sind auch in unserem Wiki zu finden:
\url{http://redmine.rdmr.ch/redmine/projects/gftprototype/wiki/Sprint_4}

\subsubsection{Hauptaufgaben / Fokussierung im Sprint}
\begin{itemize}
	\item Finalisierung aller Arbeiten
	\begin{itemize}
		\item Fertiggestellung des Use Cases \emph{FixMyStreet}
		\item Abschluss der Dokumentation
		\item Erstellung des Posters A0
	\end{itemize}
\end{itemize}

\subsubsection{Ziele}
\begin{itemize}
	\item Finalisierung des \emph{FixMyStreet} Use Cases
	\begin{itemize}
		\item Bugs / Usability-Probleme beheben
		\item Live-Daten anzeigen
		\item Berechtigungen für Benutzer
	\end{itemize}
	
	\item Finalisierung Dokumentation der Ergebnisse
	\begin{itemize}
		\item Converter-Build
		\item Use Cases / Examples
		\item Google Fusion Tables allgemein
	\end{itemize}
\end{itemize}

\subsubsection{Abgabe / Deliverables}
\begin{itemize}
	\item Fertige Dokumentation
	\item A0 Poster
	\item Vollständige Arbeit (Source Code usw.)
\end{itemize}

Nach diesem Sprint ist die Abgabe der kompletten Arbeit fällig. Dies bedeutet, dass zu diesem Zeitpunkt alle Use Cases lauffähig sind und die Dokumentation abschlossen ist. 

\section{Projektmonitoring}
\todo[inline]{Projektmonitoring schreiben}

\includegraphics[scale=0.5]{images/overall_stories_gantt_chart.png}



% -----------------------------------------
% FOOT
% -----------------------------------------
\part{Anhänge}

% Glossar
% Beispiel Glossareintrag
\newglossaryentry{computer}
{ name=computer,
  description={is a programmable machine that receives input,
               stores and manipulates data, and provides
               output in a useful format}
}

\todo[inline]{Glossar funktioniert nicht korrekt}
\gls{computer}
\printglossaries
% Referenz auf Glossar aus Inhaltsverzeichnis
\addcontentsline{toc}{chapter}{Glossar}

\chapter{Literatur}
\label{sec:Literatur}
\textsc{Jan Tschichold}:\textit{ Geschichte der Schrift in
  Bildern}.\par \hspace{9.0 pt} Hamburg, 1961.


\end{document}