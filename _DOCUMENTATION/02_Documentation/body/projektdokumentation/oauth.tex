\section{Authentifizierung mit OAuth}
\label{oauth}

\gls{OAuth} ist das Mittel der Wahl wenn es um die Authentifizierung geht im Zusammenhang mit Google Fusion Tables. Dies ist notwendig für den Zugriff auf private Tabellen und ganz allgemein beim schreibenden Zugriff.
Das Prinzip ist sehr einfach: Eine Applikation möchte Zugriff auf die Tabellen eines Benutzers. Der Benutzer möchte lieber nicht seinen Benutzernamen und sein Passwort angeben, da die Applikation damit grundsätzlich alles mit einem Account machen könnte. Es gilt also einerseits den Zugriff für die Applikation einzuschränken, gleichzeitig aber eine Authentifizierung möglich zu machen.

Bei \gls{OAuth} sind 3 Parteien involviert:
\begin{itemize}
\item Benutzer, welcher eine Applikation nutzen möchte (\emph{User})
\item Applikation die Zugriff auf eine Ressource des Benutzers haben will (\emph{Consumer})
\item Anbieter, welcher Ressourcen und Authentifizierung bietet. (\emph{Service Provider})
\end{itemize}

Sowohl der User wie auch der Consumer vertrauen dem Service Provider.

\subsection{Ablauf der Authentifizierung}
\begin{figure}[!ht]
	\centering
	\includegraphics[width=\textwidth]{images/oauth/oauth-user}
	\caption{Endbenutzer authentifiziert sich mit \gls{OAuth}}
	\label{oauth-user}
\end{figure}
Wenn der Benutzer das erste mal die gewünschte Applikation nutzen will, stellt diese fest, dass der Benutzer noch nicht authentifiziert ist. Zu diesem Zweck leitet die Applikation den Benutzer weiter zum Service Provider. Auf der Seite des Service Providers gibt der Benutzer seinen Benutzernamen und sein Passwort ein. Daraus wird das sogenannte \emph{Zugriffs-Token} generiert. Dieses Token wird anschliessend der Applikation und dem Benutzer mitgeteilt, fortan dient es dazu den Benutzer zu authentifizieren.

Die Applikation kann nun im festgelegten Rahmen direkt mit dem Token beim Service Provider auf die geschützten Ressourcen zugreifen.

\subsection{OAuth Szenarien}
Es gibt grundsätzlich verschiedene Methoden bei \gls{OAuth}, je nachdem welches Ziel eine Applikation erfolgt. Auch wenn sich der obige Ablauf immer ähnelt, gibt es Unterschiede im Detail.

\subsubsection{Clientseitiges OAuth}
Das clientseitige \gls{OAuth} entspricht dem oben beschriebenen Ablauf, ein Benutzer ist aktiv, führt die Authentifizierung durch und überlässt dann die weiteren Zugriffe der Applikation.

\subsubsection{Serverseitiges OAuth}
Beim serverseitigen \gls{OAuth} spielt der Endbenutzer eine untergeordnete Rolle, da er in den ganzen Authentifizierung keine Rolle spielt. Hier geht es darum, dass sich eine Applikation gegenüber einem Service Provider authentifiziert um auf ihre eigene Dienste zuzugreifen. Dies ist z.B. dann der Fall wenn eine Applikation als Datenbank eine \gls{Cloud}-Datenbank benutzt.

\subsection{OAuth in Google Fusion Tables}
Wir standen vor der Herausforderung die Authentifizierung für die GFT nutzbar zu machen. Für den Zugriff auf Tabellen eines Benutzers funktionierte dies ausgezeichnet, wir haben dies mit einer Beispielapplikation ausprobiert. Die Dokumentation dazu ist sehr gut\footnote{\url{https://developers.google.com/fusiontables/docs/articles/oauthfusiontables}}, so dass wenn die Konzepte klar sind mit Hilfe einer Client Library\footnote{\url{https://developers.google.com/accounts/docs/OAuth2\#libraries}} sehr einfach eine Authentisierung erstellt werden kann.

\subsubsection{FixMyStreet Use Case}
Für den Use Case FixMyStreet (siehe Kapitel \ref{fixmystreet}) waren wir jedoch vor eine neue Hürde gestellt: Wir haben GFT explizit als Applikationsdatenbank ausgelegt, so dass die Tabelle eines Benutzers alle Daten der Applikation hält. Die Applikation soll für den Endbenutzer ohne Authentifizierung benutzbar sein.

Google stellt für diesen Fall seit März 2012 sogenannte \emph{Service Accounts}\footnote{Google Developer Blog - Service Accounts have arrived: \url{http://googledevelopers.blogspot.com/2012/03/service-accounts-have-arrived.html}} zur Verfügung. Sie sind ein Beispiel für das serverseitige \gls{OAuth}.


\begin{figure}[!ht]
	\centering
	\includegraphics[width=\textwidth]{images/oauth/oauth-serviceaccount}
	\caption{Applikation verwendet einen Service Account zur Authentisierung. Der Endbenutzer merkt davon nichts.}
	\label{oauth-serviceaccount}
\end{figure}

Die Dokumentation dieses neuen Features ist leider so gut wie inexistent. Gerade auch für Google Fusion Tables fehlt ein entsprechened Service in den Client Libraries.

Die Idee bei den Service Accounts ist, dass dieser spezielle Account sich beim Service Provider anmeldet und zwar mit Hilfe eines verschlüsselten Requests. Beim erstellen des Accounts wird eine Schlüsselpaar generiert, mit dem privaten Schlüssel können dann solch verschlüsselte Requests erstellt werden.

\subsubsection{OAuth in der GftLib}
Da sich für die Service Accounts die Token nur via Server holen lassen, haben wir einen Web Service erstellt, welcher jeweils ein gültiges Token über eine \gls{JSONP}-Schnittstelle zur Verfügung stellt. Mit dem so erworbenen Token, kann die JavaScript Bibliothek GftLib sich dann bei Google Fusion Tables authentifizieren. Durch die Berechtigungen auf einer Tabelle bzw. durch den Einsatz von Views können die Möglichkeiten, welches das Token bietet, eingeschränkt werden.





