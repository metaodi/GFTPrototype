\chapter{Use Case 1: WorldData}
\label{worlddata}

\begin{center}
\includegraphics[scale=0.8]{images/usecase1-worlddata/worlddata-icon_with_gloss}

{\large \textbf{\url{http://worlddata.rdmr.ch/}}}

\vspace{1cm}

\includegraphics[width=0.8\textwidth]{images/usecase1-worlddata/screenshots/worlddata-screenshots-population_1960}
\end{center}

\section{Einführung}
Im ersten Use Case geht es hauptsächlich um die Anzeige grosser Datenmengen auf der Karte. Dazu importieren wir bestehende Datenbestände in die Google Fusion Tables und visualisieren diese mittels FusionTablesLayer des Google Maps \gls{API} (siehe Abschnitt \ref{gmap-api-fusiontableslayer}) auf der Karte.

\subsection{Ziel}
Es sollen verschiedene historische Länderdaten auf einer Weltkarte angezeigt werden. Die Daten sind pro Jahr und Thema unterteilt. Über eine Zeitachse soll es möglich sein die Daten der verschiedenen Jahre zu selektieren. Eine solche Darstellung kann beispielsweise dabei helfen Zusammenhänge zwischen verschiedenen Themenbereichen zu finden.

\subsection{Vorgehen}
\label{worlddata-vorgehen}
Um die Daten pro Land zu visualisieren, wurden zuerst die Landesgrenzen als Geometrie-Datensätze in eine Fusion Table importiert. In einer anderen Tabelle wurden dann die historischen Daten, unterteilt nach Land und Jahr, geladen. Aus diese beiden Tabellen entstand dann per \gls{Merge}-Funktion (siehe Abschnitt \ref{merge-table}) eine sogenannte Merged Table, welche die Daten der beiden zugrundeliegenden Tabellen kombiniert. Diese neue Tabelle konnte dann mittels FusionTablesLayer auf der Karte dargestellt werden.

\section{Analyse}
Die Idee des Use Cases stammt von der Webapplikation \emph{Public Data}\footnote{\url{http://www.google.ch/publicdata/}} von Google. Darin lassen sich bereits öffentliche Daten aus verschiedensten Quellen über das Google Chart \gls{API}\footnote{\url{https://developers.google.com/chart/}} in verschiedenen Diagrammen darstellen.

Als Datenquelle für diesen Use Case wurde der Datenkatalog der Weltbank\footnote{\url{http://data.worldbank.org/}} verwendet. Darin finden sich länderspezifische Daten aufgeteilt in über 7000 Themenbereiche. Diese lassen sich als \gls{XML}- oder Excel-Datei herunterladen. Für unseren Import verwendeten wir die Excel-Dateien, welche wir vorbereitet und als \gls{CSV}-Dateien gespeichert haben. Diese konnten wir anschliessend in Google Fusion Tables importieren.

Die Landesgrenzen wurden als \gls{KML}-Datei von einer inoffiziellen Google Earth Library-Webseite\footnote{\url{http://www.gelib.com/world-borders.htm}} bezogen. 

\section{Design}

\subsection{UseCases}
Folgende Anwendungsfälle sollen von der Applikation abgedeckt werden.

\begin{figure}[H]
	\centering
	\includegraphics[width=0.6\textwidth]{images/usecase1-worlddata/uml/worlddata-usecasemodel}
	\caption{WorldData: UseCase Modell}
	\label{worlddata-usecasemodel}
\end{figure}

% Use Case 1: Anzeigen von Daten eines bestimmten Themas
\subsubsection{Use Case 1: Anzeigen von Daten eines bestimmten Themas}

\renewcommand{\arraystretch}{2}
\begin{longtable}{p{0.25\twocelltabwidth}p{0.75\twocelltabwidth}}
\textit{Primary Actor} & Benutzer \\ 
\textit{Stakeholders and Interests} & Benutzer: Möchte Daten zu einem bestimmten Thema auf der Karte anzeigen \\ 
\textit{Preconditions} & Webapplikation ist gestartet \\ 
\textit{Success Guarantee (Postconditions)} & \begin{itemize}[noitemsep, nosep, leftmargin=12pt, before*={\mbox{}\vspace{-\baselineskip}}, after*={\mbox{}\vspace{-\baselineskip}}]
\item Daten des gewählten Themas werden auf der Karte angezeigt
\item Legende zu den Daten wird angezeigt
\end{itemize} \\ 
\textit{Main Success Scenario} & \begin{enumerate}[noitemsep, nosep, leftmargin=12pt, before*={\mbox{}\vspace{-\baselineskip}}, after*={\mbox{}\vspace{-\baselineskip}}]
\item Benutzer wählt das Thema aus
\end{enumerate} \\ 
\textit{Frequency of Occurrence} & Tritt sehr häufig auf, da eine beliebige Anzahl von Benutzern die Webapplikation bedienen können \\ 
\end{longtable} 
\renewcommand{\arraystretch}{1.3}

% Use Case 2: Ändern des Jahres
\subsubsection{Use Case 2: Ändern des Jahres}

\renewcommand{\arraystretch}{2}
\begin{longtable}{p{0.25\twocelltabwidth}p{0.75\twocelltabwidth}}
\textit{Primary Actor} & Benutzer \\ 
\textit{Stakeholders and Interests} & Benutzer: Möchte Daten eines anderen Jahres anzeigen \\ 
\textit{Preconditions} & \begin{itemize}[noitemsep, nosep, leftmargin=12pt, before*={\mbox{}\vspace{-\baselineskip}}, after*={\mbox{}\vspace{-\baselineskip}}]
\item Webapplikation ist gestartet
\item Thema ist ausgewählt
\end{itemize} \\ 
\textit{Success Guarantee (Postconditions)} & \begin{itemize}[noitemsep, nosep, leftmargin=12pt, before*={\mbox{}\vspace{-\baselineskip}}, after*={\mbox{}\vspace{-\baselineskip}}]
\item Daten des gewählten Jahres werden auf der Karte angezeigt
\end{itemize} \\ 
\textit{Main Success Scenario} & \begin{enumerate}[noitemsep, nosep, leftmargin=12pt, before*={\mbox{}\vspace{-\baselineskip}}, after*={\mbox{}\vspace{-\baselineskip}}]
\item Benutzer wählt das gewünschte Jahr aus
\end{enumerate} \\ 
\textit{Alternative Flows} & 1a. Falls für das gewählte Jahr keine Daten vorhanden sind, ist dies ebenfalls auf der Karte ersichtlich \\ 
\textit{Frequency of Occurrence} & Tritt sehr häufig auf, da eine beliebige Anzahl von Benutzern die Webapplikation bedienen können \\ 
\end{longtable} 
\renewcommand{\arraystretch}{1.3}

\subsection{Datenbankschema}
Wie bereits im Abschnitt \ref{worlddata-vorgehen} beschrieben, verwendet die Applikation die Daten der Merged Table \emph{MERGE-ftWorldBorders\_ftWorldData} (siehe Abschnitt \ref{merge-table}). Diese beinhaltet pro Thema und Land eine Zeile mit Daten. Diese Daten sind wiederum pro Jahr in einer Spalte abgelegt. Dieses Schema ist 1:1 von den Daten der Weltbank übernommen. Durch einen \inlinecode{JOIN} über den Namen des Landes, werden die Geometriedaten dazugenommen.

\begin{figure}[H]
	\centering
	\includegraphics[width=0.8\textwidth]{images/usecase1-worlddata/uml/worlddata-datamodel}
	\caption{WorldData: Datenbankschema}
	\label{worlddata-datamodel}
\end{figure}

\subsection{Aufbau der Applikation}
Die Applikation hat eine \emph{View} welche die Anzeige der Daten regelt. Diese wiederum wird von einem \emph{Controller} gesteuert. Beide verwenden Konfigurationsparameter, welche in der \emph{Config}-Klasse gesetzt werden können. Zusätzlich verwenden beide Klassen Helfermethoden aus der \emph{Helper}-Klasse.

\begin{figure}[H]
	\centering
	\includegraphics[width=0.8\textwidth]{images/usecase1-worlddata/uml/worlddata-classdiagram}
	\caption{WorldData: Klassendiagramm}
	\label{worlddata-classdiagram}
\end{figure}

\section{Implementation}
Die Applikation wurde als Webapplikation implementiert. Für die GUI-Elemente wurde das Javascript Framework jQuery Mobile\footnote{\url{http://jquerymobile.com/}} verwendet. Dieses bietet eine sehr breite Plattform-Unterstützung.

\subsection{Systemanforderungen}
Da es sich um eine Webapplikation handelt, ist es möglich, die Applikation auf beinahe allen Geräten, welche über einem Browser verfügen, zu starten. Einschränkungen gibt es nur in den unterstützten Browsern. Eine Liste davon findet sich auf der Webseite des jQuery Mobile Frameworks: \url{http://jquerymobile.com/blog/2012/04/06/jquery-mobile-1-1-0-rc2/#platforms}

\subsection{Abhängigkeiten}
\begin{table}[H]
\centering
\begin{tabular}{|p{0.35\threecelltabwidth}|p{0.11\threecelltabwidth}|p{0.54\threecelltabwidth}|}
\hline 
\textbf{Library} & \textbf{Version} & \textbf{Verwendung} \\ 
\hline 
jQuery Mobile & 1.1.0 & GUI-Elemente \\ 
\hline 
jQuery & 1.7.1 & Basis für den Gebrauch von jQuery Mobile \\ 
\hline 
Google Maps \gls{API} & V3 & Karte mit FusionTablesLayer \\ 
\hline 
Cubiq - Add to home screen & 2.0 & Popup welches auf allen iOS Geräten erscheint \\ 
\hline 
\end{tabular}
\caption{WorldData: Abhängigkeiten}
\end{table} 

\subsection{Quellcode-Struktur}
\begin{table}[H]
\centering
\begin{tabular}{|p{0.25\twocelltabwidth}|p{0.75\twocelltabwidth}|}
\hline 
\textbf{Datei} & \textbf{Beschreibung} \\ 
\hline 
\inlinecode{images/} & Bilder der Applikation \\ 
\hline 
\inlinecode{js/} & Hier befindet sich die eigentliche Implementation der Applikation \\ 
\hline 
\inlinecode{js/Config.js} & Konfiguration der Applikation \\ 
\hline 
\inlinecode{js/Controller.js} & Controller der Applikation \\ 
\hline 
\inlinecode{js/Helper.js} & Helper-Funktionen welche von der Applikation verwendet werden \\ 
\hline 
\inlinecode{js/View.js} & Steuert die Anzeige der Applikation \\ 
\hline 
\inlinecode{lib/} & Von der Applikation verwendete Libraries \\ 
\hline 
\inlinecode{styles/} & CSS-Styles der Applikation \\ 
\hline
\inlinecode{index.html} & Startseite der Applikation \\ 
\hline
\end{tabular}
\caption{WorldData: Quellcode-Struktur}
\end{table} 

\section{Testing}
Die Applikationslogik dieses Use Cases befindet sich im Controller und zu kleinen Teilen im Helper. Für diese beiden Komponenten haben wir insgesamt zwölf Tests erstellt\footnote{Ausführbare Tests: \url{http://gft.rdmr.ch/test/js/?filter=WorldData}}.
Zusätzlich gibt es noch zwei Tests für die Konfiguration, diese dienen vor allem dazu, Änderungen der Konfiguration zu erkennen.

Die Anzeige der Daten basiert auf dem FusionTablesLayer des Google Maps \gls{API}, weshalb wir dort keine eigene Logik haben und diese somit auch nicht testen müssen.

\section{Resultate}
Das Resultat dieses Use Cases ist eine Webapplikation, welche es dem Benutzer erlaubt länderspezifische Daten zu verschiedenen Themen auf der Karte darzustellen.

\subsection{Features}
\begin{itemize}
\item Auswahl des Themas über Selectbox
\item Wahl des Jahres über Slider
\item Weitere Informationen zu den angezeigten Daten per Klick auf das gewünschte Land
\item Applikation ist dank Verwendung des jQuery Mobile Frameworks ebenfalls auch für Mobilgeräte optimiert
\end{itemize}

\subsection{Screenshots}
\begin{figure}[H]
	\centering
	\includegraphics[width=0.6\textwidth]{images/usecase1-worlddata/screenshots/worlddata-screenshots-population_1960}
	\caption{WorldData: Population im Jahr 1960}
	\label{worlddata-screenshots-population_1960}
\end{figure}

\begin{figure}[H]
	\centering
	\includegraphics[width=0.6\textwidth]{images/usecase1-worlddata/screenshots/worlddata-screenshots-population_2010}
	\caption{WorldData: Population im Jahr 2010}
	\label{worlddata-screenshots-population_2010}
\end{figure}

\subsection{Erkenntnisse zur Fusion Table-Ebene}
Die Fusion Table-Ebene von Google Maps \gls{API} eignet sich sehr gut zur Darstellung einer grossen Anzahl von Daten. Leider hindern die Einschränkungen betreffend Styles (siehe Abschnitt \ref{fusiontableslayer-styles-restrictions}) die individuelle Gestaltung der Ebene stark.

Wie wir feststellen mussten, sind diese 5 Stile sehr schnell aufgebraucht. Einer davon fällt meistens für den Standard-Stil weg. Dieser wird angewendet, wenn die entsprechende Zeile aus der Tabelle auf keine Bedingung der restlichen Stile passt. So bleiben lediglich noch 4 Stile übrig für alle Elemente die man speziell hervorheben möchte.

Sobald man dann auch noch verschiedene Geometrie-Typen in der Tabelle abgelegt hat (Punkt, Linie, Fläche), muss man sich gut überlegen, für welche Elemente man wirklich einen speziellen Stil anwenden will.

\section{Weiterentwicklung}
Da es sich bei der Applikation lediglich um einen Prototypen handelt, bietet diese natürlich noch ein grosses Potential zur Weiterentwicklung. Hier eine Auflistung möglicher Features, welche noch implementiert werden könnten:

\begin{itemize}
\item Weitere Themen in die Datenbank importieren
\item Auswahl der Länder für welche Daten angezeigt werden sollen
\item Anzeige einer Rangliste der 10 Länder mit den höchsten Werten des gewählten Themas
\item Direkter Vergleich verschiedener Länder
\end{itemize}


\section{Benutzerdokumentation}
\subsection{Importieren der Daten in Google Fusion Tables}

\subsubsection{Landesgrenzen}
\label{landesgrenzen}
Die Landesgrenzen liegen als \gls{KML}-Datei vor. Diese beinhaltet alle Länder mit ihren Grenzen definiert als Polygone.

\begin{figure}[H]
	\centering
	\includegraphics[width=0.9\textwidth]{images/usecase1-worlddata/documentation/worlddata-worldborders_kml}
	\caption{WorldData: Landesgrenzen liegen als KML-Datei vor}
	\label{worlddata-worldborders_kml}
\end{figure}

\begin{enumerate}
\item Google Drive öffnen (\url{https://drive.google.com/}) und sich mit seinem Google Account einloggen
\item Erstellen > Mehr > Fusion Table (Beta) \\ \includegraphics[width=0.5\textwidth]{images/usecase1-worlddata/documentation/worlddata-worldborders_import1}
\item Es öffnet sich die neue Fusion Table mit dem Dialog zum Importieren von bestehenden Daten
\item Im Tab \emph{From this computer} die lokal gespeicherte Datei auswählen > Next
\item Die Daten werden automatisch in passende Spalten eingeteilt
\item Abschliessend muss der Tabelle noch einen Namen gegeben werden
\item Mit \emph{Finish} werden die Daten dann importiert
\end{enumerate}

Die Tabelle sollte nun folgendermassen aussehen:

\begin{figure}[H]
	\centering
	\includegraphics[width=\textwidth]{images/usecase1-worlddata/documentation/worlddata-worldborders_import_done}
	\caption{WorldData: Landesgrenzen erfolgreich als Fusion Table importiert}
	\label{worlddata-worldborders_import_done}
\end{figure}

\subsubsection{Daten}
Die Daten liegen als \gls{CSV}-Datei vor. Diese beinhaltet folgende Spalten:
\begin{itemize}
\item Country Name
\item Country Code
\item Indicator Name
\item Indicator Code
\item 1960
\item 1961
\item ...
\end{itemize}

\begin{figure}[H]
	\centering
	\includegraphics[width=0.9\textwidth]{images/usecase1-worlddata/documentation/worlddata-data_csv}
	\caption{WorldData: Daten liegen als CSV-Datei vor}
	\label{worlddata-data_csv}
\end{figure}

Das Vorgehen für den Import der Daten ist dasselbe wie bei den Landesgrenzen (siehe Abschnitt \ref{landesgrenzen}).

Nach dem Import der Daten sollte die Tabelle folgendermassen aussehen:

\begin{figure}[H]
	\centering
	\includegraphics[width=\textwidth]{images/usecase1-worlddata/documentation/worlddata-data_import_done}
	\caption{WorldData: Daten erfolgreich als Fusion Table importiert}
	\label{worlddata-data_import_done}
\end{figure}

\subsubsection{Tabellen mergen}
Sind beide Fusion Tables erstellt müssen diese zusammengefügt werden, um sie schlussendlich als einen einzelnen Layer auf der Karte darzustellen. Dazu bietet Google Fusion Tables die \emph{\gls{Merge}}-Funktion an. 

\begin{enumerate}
\item Die Tabelle mit den Landesgrenzen öffnen
\item Im Menü \emph{\gls{Merge}} auswählen \\ \includegraphics{images/usecase1-worlddata/documentation/worlddata-merge1}
\item Es öffnet sich ein Popup, welches durch den Vorgang führt
\item In der 1. Spalte wählt man die Spalte aus, über welche die beiden Tabellen verbunden werden sollen. In unserem Fall ist dies die Spalte mit den Namen der Länder.
\item In der 2. Spalte muss man zuerst die andere Tabelle auswählen und dann ebenfalls die Spalte, in welcher die Namen der Länder gespeichert sind.
\item Schlussendlich muss der neuen Tabelle ein Namen gegeben werden
\item Mit einem Klick auf \emph{Merge tables} wird die neue Tabelle mit den zusammengefügten Daten erstellt. \\ \includegraphics[width=0.7\textwidth]{images/usecase1-worlddata/documentation/worlddata-merge2}
\end{enumerate}

Die zusammengeführte Tabelle sollte nun folgendermassen aussehen:

\begin{figure}[H]
	\centering
	\includegraphics[width=\textwidth]{images/usecase1-worlddata/documentation/worlddata-merge_done}
	\caption{WorldData: Merge der Landesgrenzen- und Daten-Tabelle}
	\label{worlddata-merge_done}
\end{figure}

\subsubsection{Merge-Tabelle für die Verwendung mit FusionTablesLayer vorbereiten}
Um die Tabelle nun als FusionTablesLayer verwenden zu können, muss diese als \emph{öffentlich} markiert werden. Dazu klickt man bei der geöffneten Tabelle auf den \emph{Share}-Button in der linken oberen Ecke.

\begin{figure}[H]
	\centering
	\includegraphics[width=0.7\textwidth]{images/usecase1-worlddata/documentation/worlddata-prepare_fusiontableslayer1}
	\caption{WorldData: Freigabe der Merge-Tabelle}
	\label{worlddata-prepare_fusiontableslayer1}
\end{figure}

Im Dialogfenster wählt man unter \emph{Visibility options} den Eintrag \emph{Public} aus.

\begin{figure}[H]
	\centering
	\includegraphics[width=0.6\textwidth]{images/usecase1-worlddata/documentation/worlddata-prepare_fusiontableslayer2}
	\caption{WorldData: Freigabeeinstellungen der Merge-Tabelle}
	\label{worlddata-prepare_fusiontableslayer2}
\end{figure}

Als letzten Schritt muss man sich noch die eindeutige ID der Tabelle merken. Dazu wählt man im Menü  \emph{File > About} und kopiert sich die angezeigte \emph{Encrypted ID} im geöffneten Dialog.

\begin{figure}[H]
	\centering
	\includegraphics[width=0.4\textwidth]{images/usecase1-worlddata/documentation/worlddata-prepare_fusiontableslayer3}
	\caption{WorldData: ID der Merge-Tabelle finden}
	\label{worlddata-prepare_fusiontableslayer3}
\end{figure}

\begin{figure}[H]
	\centering
	\includegraphics[width=0.7\textwidth]{images/usecase1-worlddata/documentation/worlddata-prepare_fusiontableslayer4}
	\caption{WorldData: Dialog mit ID der Merge-Tabelle}
	\label{worlddata-prepare_fusiontableslayer4}
\end{figure}

\subsection{Konfiguration der Applikation}
Um die eben erstellte Fusion Table nun in der WorldData-Applikation zu verwenden muss das Konfigurations-File der Applikation folgendermassen angepasst werden.

\subsubsection{/js/Config.js}
\begin{longtable}{|p{0.4\threecelltabwidth}|p{0.1\threecelltabwidth}|p{0.5\threecelltabwidth}|}
\hline 
\textbf{Parameter} & \textbf{Typ} & \textbf{Beschreibung} \\ 
\hline 
\inlinecode{\$.fusiontable.id} & \inlinecode{string} & Öffnetliche ID der zu verwendenden Fusion Table \\ 
\hline 
\inlinecode{\$.fusiontable.field} & \inlinecode{string} & Spaltenname welcher die Landesgrenzen speichert

\textit{(ACHTUNG: Gross-/Kleinschreibung relevant)} \\ 
\hline 
\inlinecode{\$.fusiontable.typeField} & \inlinecode{string} & Spaltenname aus welchem die Themen gelesen werden sollen

\textit{(ACHTUNG: Gross-/Kleinschreibung relevant)} \\ 
\hline 
\inlinecode{\$.fusiontable.types} & - & Konfiguration der verschiedenen Themen. Für jedes Thema muss ein neuer Block nach folgendem Schema erstellt werden:

\lstset{language=JavaScript}
\begin{lstlisting}
'<TYPE>': {
	name: '<TYPE.name>',
	styleBoundaries: {
		low: <TYPE.styleBoundaries.low>,
		medium: <TYPE.styleBoundaries.medium>,
		high: <TYPE.styleBoundaries.high>
	}
}
\end{lstlisting} \\ 
\hline 
\inlinecode{TYPE} & \inlinecode{string} & ID des Themas \\ 
\hline 
\inlinecode{TYPE.name} & \inlinecode{string} & Bezeichnung des Themas \\ 
\hline 
\inlinecode{TYPE.styleBoundaries.low} & \inlinecode{int} & Obere Grenze für tiefe Werte \\ 
\hline 
\inlinecode{TYPE.styleBoundaries.medium} & \inlinecode{int} & Obere Grenze für mittlere Werte \\ 
\hline 
\inlinecode{TYPE.styleBoundaries.high} & \inlinecode{int} & Obere Grenze für hohe Werte \\ 
\hline 
\caption{WorldData: Konfigurationsparameter}
\end{longtable}

\subsection{Starten der Applikation}
Beim Starten der Applikation sollten nun alle Themen der angegebenen Fusion Table in das Auswahlfeld \emph{Ebene auswählen...} geladen werden.

\begin{figure}[H]
	\centering
	\includegraphics[width=0.6\textwidth]{images/usecase1-worlddata/documentation/worlddata-application_start}
	\caption{WorldData: Auswahlliste der Themen}
	\label{worlddata-application_start}
\end{figure}