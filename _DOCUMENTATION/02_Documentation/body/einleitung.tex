\chapter{Einleitung}
In unserer Arbeit untersuchen wir die Möglichkeiten welche die Clouddatenbank Google Fusion Tables bietet. Es sollen einige Prototypen für verschiedenste Anwendungsfälle im GIS-Bereich erstellt werden, welche das Potential der Datenbank aufzeigen.

\section{Problemstellung}
Die Aufgabenstellung stammt von der GEOINFO AG (www.geoinfo.ch), welche massgeschneiderte GIS-Softwarelösungen für ihre Kunden entwickelt.

Für solche Unternehmen wird es nach und nach schwieriger sich auf dem Markt zu beweisen, da bereits viele cloudbasierte GIS-Lösungen sehr günstig oder gar kostenlos erhältlich sind. Durch die Prototypen soll ersichtlich gemacht werden, welche Anwendungsfälle von bestehenden proprietären GIS-Sytemen bereits mit Google Fusion Tables realisierbar wären und welchen Aufwand dies darstellen würde.

\section{Aufgabenstellung}
Im Rahmen dieser Arbeit sollen das Potential aber auch Einschränkungen von Google Fusion Tables für den Einsatzbereich eines öffentlichen Web GIS evaluiert werden. Es ist aufzuzeigen, welche der typischen Anwendungsfälle, wie sie in aktuellen Web GIS Lösungen (z.B. www.geoportal.ch, www.stadtplan.stadt-zuerich.ch) implementiert sind, auf Basis von Google Fusion Tables und Google Maps realisiert werden könnten. Eine Auswahl dieser Grundfunktionen ist anhand eines Prototypen zu implementieren. Die Zielgruppe sind demnach GIS-Sachbearbeiter.

\section{Ziele}
In der Aufgabenstellung der Arbeit wurden folgende Ziele definiert:
\begin{itemize}
\item Evaluation von Google Fusion Talbes in Kombination mit Google Maps u.a. mit Blick auf deren Funktionalität, Anwendbarkeit, Zuverlässigkeit und Performance
\item Entwurf und Dokumentation einer GIS-Architektur, welche einerseits Cloud Services (am Beispiel von Fusion Talbes) andererseits die Geodaten- und Serviceinfrastruktur einer Organisation integriert oder migriert.
\item Analyse verschiedener Use Cases wobei einer davon als Prototyp einer Webapplikation (voll) implementiert werden soll. 
\item Implementierung und Bewertung von verschiedenen cloudbasierten GIS-Prototypen unter Verwendung der Google Fusion Tables API
\begin{itemize}
	\item Prototyp(en) aus Use Case-Evaluation (oben). Dabei sollen v.a. auch die Geometrietypen Linestring und Polygon berücksichtigt werden.
	\item Prototyp einer Datenerfassung/Verwaltung am Beispiel eines Point-of-Interest (POI) Layers mit grossen Datenmengen
\end{itemize}
\item Prototyping für zukünftige (GIS-) Kollaborationsplattformen. Es soll aufgezeigt werden, wie sich bestehende Konzepte verbessern lassen oder weiterentwickeln könnten (z.B. www.mysg.ch/locations oder http://ch.tilllate.com/de/locations).
\end{itemize}

\section{Rahmenbedingung}
\begin{itemize}
\item Es gelten die Rahmenbedingungen, Vorgaben und Termine der HSR
\item Die Projektabwicklung orientiert sich an einer iterativen, agilen Vorgehensweise. Als Vorgabe dient dabei Scrum, wobei bedingt durch das kleine Projektteam gewisse Vereinfachungen vorgenommen werden. Meilensteine werden bezüglich Termin und Inhalt mit dem verantwortlichen Dozenten und dem Projektpartner vereinbart.
\item Die Kommunikation in der Projektgruppe, in der Dokumentation und an den Präsentationen erfolgt in Deutsch.
\item Eine Prototypen-Website ist in HTML/JavaScript zu implementieren und sollte auf verschiedenen Plattformen lauffähig sein.
\end{itemize}

\section{Vorgehen}
Die Arbeit umfasst zwei Themenbereiche. Einen theoretischen Teil in dem untersucht wird, inwiefern Google Fusion Tables eine Konkurrenz für bestehende proprietäre GIS-Lösungen darstellt. Als zweiten Teil sollen verschiedene Anwendungsfälle mit Google Fusion Tables als Prototypen nachgebaut werden, wobei einer davon als mobile Webapplikation ausprogrammiert werden soll.

\subsection{Potential von Google Fusion Tables}
In einem theoretischen Teil sollen Erkenntnisse gewonnen werden, inwiefern sich bestehende GIS-Lösungen mit Google Fusion Tables ablösen lassen.

Dies soll am Beispiel eines Migrationsszenarios festgestellt werden. Die bestehenden GIS-Daten sollen möglichst einfach Exportiert und anschliessend in eine Google Fusion Tables-Datenbank importiert werden. Dafür werden die geeigneten Formate zur Übertragung der Daten evaluiert.

Da sich die Datenstrukturen von bestehenden Lösungen stark unterscheiden können, ist es nicht das Ziel eine Schritt-für-Schritt Anleitung zu erstellen. Primär geht es darum verschiedene Migrationslösungen zu entwickeln und miteinander zu vergleichen.

Dadurch soll das gegenwärtige Potential von Google Fusion Tables abgeschätzt werden. Diese Erkenntnis soll GIS-Lösungsprovidern einen Überblick verschaffen, inwiefern Google Fusion Tables bereits als einen möglichen Konkurrenten angesehen werden muss.

\subsection{Prototypen}
Der beschriebene theoretische Teil soll anschliessend durch verschiedene Prototypen belegt werden. Es soll versucht werden mehrere Standard-Anwendungsfälle im GIS-Bereich mittels Google Fusion Tables zu realisieren. Diese Prototypen sollen hauptsächlich als Webclients entwickelt werden.

Ein Anwendungsfall soll zusätzlich als vollwertige Webapplikation für Mobilgeräte ausprogrammiert werden.

\section{Aufbau der Arbeit}

\section{Stand der Technik}

\section{Vision/Umsetzungskonzept}

\section{Resultate der Arbeit}

\section{Schlussfolgerungen}

\section{Ausblick}