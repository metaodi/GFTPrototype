% Glossar
\newglossaryentry{Geocodierung} {
	name = Geocodierung,
	description = {Bei der Geocodierung wird eine Zeichenkette (Adresse, Namen) einer geografischen Position zugewiesen}
}

\newglossaryentry{KML} {
	name = KML,
	description = {Keyhole Markup Languagen\footnote{\url{{http://de.wikipedia.org/wiki/Keyhole_Markup_Language}}}}
}

\newglossaryentry{CSV} {
	name = CSV,
	description = {Comma-separated values\footnote{\url{{http://de.wikipedia.org/wiki/CSV_(Dateiformat)}}}}
}

\newglossaryentry{XML} {
	name = XML,
	description = {Extensible Markup Language\footnote{\url{{http://de.wikipedia.org/wiki/Extensible_Markup_Language}}}}
}

\newglossaryentry{API} {
	name = API,
	description = {Application Programming Interface\footnote{\url{{http://de.wikipedia.org/wiki/Programmierschnittstelle}}}}
}

\newglossaryentry{GIS} {
	name = GIS,
	description = {Geoinformationssysteme\footnote{\url{{http://de.wikipedia.org/wiki/Geoinformationssystem}}}}
}

\newglossaryentry{SaaS} {
	name = SaaS,
	description = {Software as a Service\footnote{\url{{http://de.wikipedia.org/wiki/Software_as_a_Service}}}}
}

\newglossaryentry{REST} {
	name = REST,
	description = {Representational State Transfer\cite{rest} ist ein Programmierparadigma, welches besagt, dass sich der Zustand einer Webapplikation als Ressource in Form einer URL beschreiben lässt. Auf eine solche Ressourcen können folgende Befehle angewendet werden: \inlinecode{GET}, \inlinecode{POST}, \inlinecode{PUT},\inlinecode{PATCH}, \inlinecode{DELETE}, \inlinecode{HEAD} und \inlinecode{OPTIONS}. HTTP ist ein Protokoll welches REST implementiert}
}

\newglossaryentry{AJAX} {
	name = AJAX,
	description = {Asynchronous JavaScript and XML\footnote{\url{http://de.wikipedia.org/wiki/Ajax_(Programmierung)}} beschreibt die Möglichkeit via JavaScript Daten von einem Server nachzuladen}
}

\newglossaryentry{JSONP} {
	name = JSONP,
	description = {JavaScript Object Notation with Padding\footnote{\url{http://de.wikipedia.org/wiki/JavaScript_Object_Notation\#JSONP_.28JSON_mit_Padding.29}} beschreibt die Möglichkeit über Domaingrenzen hinweg Daten zu laden. Es ist eine möglich Lösung zu der Same-Origin-Policy Problematik\cite{sop}}
}

\newglossaryentry{OAuth} {
	name = OAuth,
	description = {OAuth\footnote{\url{http://de.wikipedia.org/wiki/Oauth}} ist ein offenes Protokoll, das eine standardisierte, sichere API-Autorisierung erlaubt}
}

\newglossaryentry{AntTarget} {
	name = Ant Target,
	description = {Das Buildautomatisierungs-Tool Ant\footnote{\url{http://ant.apache.org/}} nennt die einzelnen Schritte eines Builds \emph{Target}. Targets sind eine Sammlung von semantisch zusammengehörigen Tasks\footnote{\url{http://ant.apache.org/manual/targets.html}}. Sie können Abhängigkeiten zu anderen Targets aufweisen, welche dann als Vorbedingung zuerst ausgeführt werden}
}

\newglossaryentry{Cloud} {
	name = Cloud,
	description = {Als Cloud oder Cloud-Computing (\emph{Wolke}) bezeichnet man die Gesamtheit aller Dienste, welche ortsunabhängig im Internet angeboten werden. Dies können zum Beispiel Datenspeicher, Server oder Datenbanken oder schlicht Rechenleistung sein. Der grosse Vorteil der Cloud ist, dass sie sehr leicht skalierbar ist, so dass man  die Leistungen dynamisch an den Bedarf angepassen kann.\cite{cloud}}
}

\newglossaryentry{Merge} {
	name = Merge,
	description = {Bei einem \emph{Merge} werden 2 Dinge zusammengeführt, im Fall von Google Fusion Tables können 2 Tabellen mit einem \emph{Merge} zu einer sogenannten \emph{Merged Table} zusammengeführt werden.}
}

\newglossaryentry{HeadlessBrowser} {
	name = headless Browser,
	description = {Bei einem \emph{headless Browser} handelt es sich um einen Browser, welcher ohne grafische Benutzeroberfläche auskommt. Häufig werden sie dazu verwendet, um Serverjobs, welche auf den Browser angewiesen sind, auszuführen.}
}